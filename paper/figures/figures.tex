% ============================================================
% paper/figures/figures.tex
% ============================================================
% Figures live in repository root: ../figures/*.pdf
% Included from paper/main.tex via % ============================================================
% paper/figures/figures.tex  (FULL REPLACEMENT)
% ============================================================
% Figures live in repository root: ../figures/*.pdf
% Included from paper/main.tex via % ============================================================
% paper/figures/figures.tex  (FULL REPLACEMENT)
% ============================================================
% Figures live in repository root: ../figures/*.pdf
% Included from paper/main.tex via % ============================================================
% paper/figures/figures.tex  (FULL REPLACEMENT)
% ============================================================
% Figures live in repository root: ../figures/*.pdf
% Included from paper/main.tex via \input{figures/figures}.
% ============================================================

\section*{Figures}

All figures included in this paper are generated from the frozen Evidence
Pack and are provided as external PDF files.
They are included here without modification to preserve numerical and visual
integrity.
Unless stated otherwise, figures illustrating scaling behavior are to be
interpreted as numerical consistency checks of analytic constructions or
analysis pipelines, not as independent inference evidence.

Throughout this section, it is essential to distinguish between
\emph{inference-only regimes} (no self-consistent closure; shot-noise--type
scalings such as $\Flux^{-1/2}$ may arise depending on the task/model) and
\emph{closure-limited constructions}, where estimator uncertainty feeds back
into the effective observation window, producing different exponents.

\begin{figure}[!ht]
\centering
\includegraphics[width=0.98\linewidth]{../figures/fig1_overview.pdf}
\caption{DRT concept overview: distinguishability accumulates at a finite rate
determined by the measurement channel, noise, and statistics.
Operational resolution bounds follow from rate-limited inference and may depend
qualitatively on whether the task is inference-only or involves a
self-consistent closure.}
\label{fig:overview}
\end{figure}

\begin{figure}[!ht]
\centering
\includegraphics[width=0.88\linewidth]{../figures/fig2_master_inequality_cartoon.pdf}
\caption{Schematic master inequality: ideal information accumulation versus an
upper bound suppressed by noise and finite-rate observation.
Shot-noise--type behavior (e.g.\ $\Flux^{-1/2}$) can arise in inference-only
settings without closure, while alternative exponents emerge once additional
self-consistency constraints are imposed.}
\label{fig:master}
\end{figure}

\begin{figure}[!ht]
\centering
\includegraphics[width=0.88\linewidth]{../figures/fig3_phi_scaling.pdf}
\caption{Poisson-limited diffusion localization:
$\deltat \propto \Flux^{-1/3}$ obtained as a
\emph{self-consistent fixed-point construction} (Class~0A; non-inference).
Here the estimator variance depends on the observation window, which is itself
determined self-consistently.
The Monte Carlo curve serves as a numerical consistency check of this closure,
not as inference evidence.
By contrast, inference-only formulations that do not include such a closure can
exhibit shot-noise--type behavior (e.g.\ $\Flux^{-1/2}$), depending on the
posed decision task and channel model.}
\label{fig:phi_scaling}
\end{figure}

\begin{figure}[!ht]
\centering
\includegraphics[width=0.88\linewidth]{../figures/fig4_ou_gamma_bound.pdf}
\caption{OU-process / continuous monitoring bound:
correlation-driven Fisher information yields an operational limit on parameter
resolution.
The bound illustrates rate-limited inference behavior rather than a universal
scaling exponent.}
\label{fig:ou}
\end{figure}

\begin{figure}[!ht]
\centering
\includegraphics[width=0.88\linewidth]{../figures/fig5_ramsey_phase_diagram.pdf}
\caption{Ramsey meeting-point (MP):
inference-limited phase scale versus a dynamical scale.
The crossing identifies an operational boundary set by competing rates, not a
power-law scaling.}
\label{fig:ramsey}
\end{figure}

\begin{figure}[!ht]
\centering
\includegraphics[width=0.88\linewidth]{../figures/fig6_mzi_visibility.pdf}
\caption{Mach--Zehnder meeting-point (MP):
inference-limited phase scale versus a geometric scale under visibility and noise
constraints.
As in the Ramsey case, the figure illustrates a trade-off rather than a scaling
law.}
\label{fig:mzi}
\end{figure}

\begin{figure}[!ht]
\centering
\includegraphics[width=0.88\linewidth]{../figures/fig7_noise_suppression_bound.pdf}
\caption{Noise suppresses Fisher information accumulation:
ideal linear growth is replaced by a bounded envelope under exponential
suppression, emphasizing the role of the observation channel in rate-limited
inference.}
\label{fig:noise}
\end{figure}

\begin{figure}[!ht]
\centering
\includegraphics[width=0.88\linewidth]{../figures/fig8_phi_slope_hist.pdf}
\caption{Multi-seed slope statistics for the diffusion $\Flux$-scaling experiment.
The distribution is consistent with the fixed-point construction exponent
$-1/3$ (Class~0A; closure-limited, non-inference) and demonstrates numerical
stability of the fitting pipeline.
It should not be interpreted as evidence against inference-only shot-noise
behavior, which can yield $\Flux^{-1/2}$ under tasks/models that do not include
self-consistent closure.}
\label{fig:phi_hist}
\end{figure}

\begin{figure}[!ht]
\centering
\includegraphics[width=0.88\linewidth]{../figures/fig9_ctrw_alpha_sweep.pdf}
\caption{CTRW $\alpha$-sweep
(Class~0B; exponent-imposed generator; non-inference):
fitted slopes versus the expected construction exponent $-1/(2+\alpha)$ across
transport exponents.
The sweep validates the generator and analysis pipeline but does not constitute
independent inference evidence.}
\label{fig:ctrw_alpha}
\end{figure}

\begin{figure}[!ht]
\centering
\includegraphics[width=0.88\linewidth]{../figures/fig10_ramsey_optimal_time.pdf}
\caption{Ramsey optimal interrogation time under decoherence:
information gain versus loss yields a finite optimum,
illustrating an operational trade-off rather than a scaling law.}
\label{fig:ramsey_opt}
\end{figure}

% ============================================================
% End of paper/figures/figures.tex
% ============================================================
.
% ============================================================

\section*{Figures}

All figures included in this paper are generated from the frozen Evidence
Pack and are provided as external PDF files.
They are included here without modification to preserve numerical and visual
integrity.
Unless stated otherwise, figures illustrating scaling behavior are to be
interpreted as numerical consistency checks of analytic constructions or
analysis pipelines, not as independent inference evidence.

Throughout this section, it is essential to distinguish between
\emph{inference-only regimes} (no self-consistent closure; shot-noise--type
scalings such as $\Flux^{-1/2}$ may arise depending on the task/model) and
\emph{closure-limited constructions}, where estimator uncertainty feeds back
into the effective observation window, producing different exponents.

\begin{figure}[!ht]
\centering
\includegraphics[width=0.98\linewidth]{../figures/fig1_overview.pdf}
\caption{DRT concept overview: distinguishability accumulates at a finite rate
determined by the measurement channel, noise, and statistics.
Operational resolution bounds follow from rate-limited inference and may depend
qualitatively on whether the task is inference-only or involves a
self-consistent closure.}
\label{fig:overview}
\end{figure}

\begin{figure}[!ht]
\centering
\includegraphics[width=0.88\linewidth]{../figures/fig2_master_inequality_cartoon.pdf}
\caption{Schematic master inequality: ideal information accumulation versus an
upper bound suppressed by noise and finite-rate observation.
Shot-noise--type behavior (e.g.\ $\Flux^{-1/2}$) can arise in inference-only
settings without closure, while alternative exponents emerge once additional
self-consistency constraints are imposed.}
\label{fig:master}
\end{figure}

\begin{figure}[!ht]
\centering
\includegraphics[width=0.88\linewidth]{../figures/fig3_phi_scaling.pdf}
\caption{Poisson-limited diffusion localization:
$\deltat \propto \Flux^{-1/3}$ obtained as a
\emph{self-consistent fixed-point construction} (Class~0A; non-inference).
Here the estimator variance depends on the observation window, which is itself
determined self-consistently.
The Monte Carlo curve serves as a numerical consistency check of this closure,
not as inference evidence.
By contrast, inference-only formulations that do not include such a closure can
exhibit shot-noise--type behavior (e.g.\ $\Flux^{-1/2}$), depending on the
posed decision task and channel model.}
\label{fig:phi_scaling}
\end{figure}

\begin{figure}[!ht]
\centering
\includegraphics[width=0.88\linewidth]{../figures/fig4_ou_gamma_bound.pdf}
\caption{OU-process / continuous monitoring bound:
correlation-driven Fisher information yields an operational limit on parameter
resolution.
The bound illustrates rate-limited inference behavior rather than a universal
scaling exponent.}
\label{fig:ou}
\end{figure}

\begin{figure}[!ht]
\centering
\includegraphics[width=0.88\linewidth]{../figures/fig5_ramsey_phase_diagram.pdf}
\caption{Ramsey meeting-point (MP):
inference-limited phase scale versus a dynamical scale.
The crossing identifies an operational boundary set by competing rates, not a
power-law scaling.}
\label{fig:ramsey}
\end{figure}

\begin{figure}[!ht]
\centering
\includegraphics[width=0.88\linewidth]{../figures/fig6_mzi_visibility.pdf}
\caption{Mach--Zehnder meeting-point (MP):
inference-limited phase scale versus a geometric scale under visibility and noise
constraints.
As in the Ramsey case, the figure illustrates a trade-off rather than a scaling
law.}
\label{fig:mzi}
\end{figure}

\begin{figure}[!ht]
\centering
\includegraphics[width=0.88\linewidth]{../figures/fig7_noise_suppression_bound.pdf}
\caption{Noise suppresses Fisher information accumulation:
ideal linear growth is replaced by a bounded envelope under exponential
suppression, emphasizing the role of the observation channel in rate-limited
inference.}
\label{fig:noise}
\end{figure}

\begin{figure}[!ht]
\centering
\includegraphics[width=0.88\linewidth]{../figures/fig8_phi_slope_hist.pdf}
\caption{Multi-seed slope statistics for the diffusion $\Flux$-scaling experiment.
The distribution is consistent with the fixed-point construction exponent
$-1/3$ (Class~0A; closure-limited, non-inference) and demonstrates numerical
stability of the fitting pipeline.
It should not be interpreted as evidence against inference-only shot-noise
behavior, which can yield $\Flux^{-1/2}$ under tasks/models that do not include
self-consistent closure.}
\label{fig:phi_hist}
\end{figure}

\begin{figure}[!ht]
\centering
\includegraphics[width=0.88\linewidth]{../figures/fig9_ctrw_alpha_sweep.pdf}
\caption{CTRW $\alpha$-sweep
(Class~0B; exponent-imposed generator; non-inference):
fitted slopes versus the expected construction exponent $-1/(2+\alpha)$ across
transport exponents.
The sweep validates the generator and analysis pipeline but does not constitute
independent inference evidence.}
\label{fig:ctrw_alpha}
\end{figure}

\begin{figure}[!ht]
\centering
\includegraphics[width=0.88\linewidth]{../figures/fig10_ramsey_optimal_time.pdf}
\caption{Ramsey optimal interrogation time under decoherence:
information gain versus loss yields a finite optimum,
illustrating an operational trade-off rather than a scaling law.}
\label{fig:ramsey_opt}
\end{figure}

% ============================================================
% End of paper/figures/figures.tex
% ============================================================
.
% ============================================================

\section*{Figures}

All figures included in this paper are generated from the frozen Evidence
Pack and are provided as external PDF files.
They are included here without modification to preserve numerical and visual
integrity.
Unless stated otherwise, figures illustrating scaling behavior are to be
interpreted as numerical consistency checks of analytic constructions or
analysis pipelines, not as independent inference evidence.

Throughout this section, it is essential to distinguish between
\emph{inference-only regimes} (no self-consistent closure; shot-noise--type
scalings such as $\Flux^{-1/2}$ may arise depending on the task/model) and
\emph{closure-limited constructions}, where estimator uncertainty feeds back
into the effective observation window, producing different exponents.

\begin{figure}[!ht]
\centering
\includegraphics[width=0.98\linewidth]{../figures/fig1_overview.pdf}
\caption{DRT concept overview: distinguishability accumulates at a finite rate
determined by the measurement channel, noise, and statistics.
Operational resolution bounds follow from rate-limited inference and may depend
qualitatively on whether the task is inference-only or involves a
self-consistent closure.}
\label{fig:overview}
\end{figure}

\begin{figure}[!ht]
\centering
\includegraphics[width=0.88\linewidth]{../figures/fig2_master_inequality_cartoon.pdf}
\caption{Schematic master inequality: ideal information accumulation versus an
upper bound suppressed by noise and finite-rate observation.
Shot-noise--type behavior (e.g.\ $\Flux^{-1/2}$) can arise in inference-only
settings without closure, while alternative exponents emerge once additional
self-consistency constraints are imposed.}
\label{fig:master}
\end{figure}

\begin{figure}[!ht]
\centering
\includegraphics[width=0.88\linewidth]{../figures/fig3_phi_scaling.pdf}
\caption{Poisson-limited diffusion localization:
$\deltat \propto \Flux^{-1/3}$ obtained as a
\emph{self-consistent fixed-point construction} (Class~0A; non-inference).
Here the estimator variance depends on the observation window, which is itself
determined self-consistently.
The Monte Carlo curve serves as a numerical consistency check of this closure,
not as inference evidence.
By contrast, inference-only formulations that do not include such a closure can
exhibit shot-noise--type behavior (e.g.\ $\Flux^{-1/2}$), depending on the
posed decision task and channel model.}
\label{fig:phi_scaling}
\end{figure}

\begin{figure}[!ht]
\centering
\includegraphics[width=0.88\linewidth]{../figures/fig4_ou_gamma_bound.pdf}
\caption{OU-process / continuous monitoring bound:
correlation-driven Fisher information yields an operational limit on parameter
resolution.
The bound illustrates rate-limited inference behavior rather than a universal
scaling exponent.}
\label{fig:ou}
\end{figure}

\begin{figure}[!ht]
\centering
\includegraphics[width=0.88\linewidth]{../figures/fig5_ramsey_phase_diagram.pdf}
\caption{Ramsey meeting-point (MP):
inference-limited phase scale versus a dynamical scale.
The crossing identifies an operational boundary set by competing rates, not a
power-law scaling.}
\label{fig:ramsey}
\end{figure}

\begin{figure}[!ht]
\centering
\includegraphics[width=0.88\linewidth]{../figures/fig6_mzi_visibility.pdf}
\caption{Mach--Zehnder meeting-point (MP):
inference-limited phase scale versus a geometric scale under visibility and noise
constraints.
As in the Ramsey case, the figure illustrates a trade-off rather than a scaling
law.}
\label{fig:mzi}
\end{figure}

\begin{figure}[!ht]
\centering
\includegraphics[width=0.88\linewidth]{../figures/fig7_noise_suppression_bound.pdf}
\caption{Noise suppresses Fisher information accumulation:
ideal linear growth is replaced by a bounded envelope under exponential
suppression, emphasizing the role of the observation channel in rate-limited
inference.}
\label{fig:noise}
\end{figure}

\begin{figure}[!ht]
\centering
\includegraphics[width=0.88\linewidth]{../figures/fig8_phi_slope_hist.pdf}
\caption{Multi-seed slope statistics for the diffusion $\Flux$-scaling experiment.
The distribution is consistent with the fixed-point construction exponent
$-1/3$ (Class~0A; closure-limited, non-inference) and demonstrates numerical
stability of the fitting pipeline.
It should not be interpreted as evidence against inference-only shot-noise
behavior, which can yield $\Flux^{-1/2}$ under tasks/models that do not include
self-consistent closure.}
\label{fig:phi_hist}
\end{figure}

\begin{figure}[!ht]
\centering
\includegraphics[width=0.88\linewidth]{../figures/fig9_ctrw_alpha_sweep.pdf}
\caption{CTRW $\alpha$-sweep
(Class~0B; exponent-imposed generator; non-inference):
fitted slopes versus the expected construction exponent $-1/(2+\alpha)$ across
transport exponents.
The sweep validates the generator and analysis pipeline but does not constitute
independent inference evidence.}
\label{fig:ctrw_alpha}
\end{figure}

\begin{figure}[!ht]
\centering
\includegraphics[width=0.88\linewidth]{../figures/fig10_ramsey_optimal_time.pdf}
\caption{Ramsey optimal interrogation time under decoherence:
information gain versus loss yields a finite optimum,
illustrating an operational trade-off rather than a scaling law.}
\label{fig:ramsey_opt}
\end{figure}

% ============================================================
% End of paper/figures/figures.tex
% ============================================================
.
% ============================================================

\section*{Figures}

All figures included in this paper are generated from the frozen Evidence
Pack and are provided as external PDF files. They are included here without
modification to preserve numerical and visual integrity.
Unless stated otherwise, figures illustrating scaling behavior are to be
interpreted as numerical consistency checks of analytic constructions or
analysis pipelines, not as independent inference evidence.

\begin{figure}[!ht]
\centering
\includegraphics[width=0.98\linewidth]{../figures/fig1_overview.pdf}
\caption{DRT concept overview: distinguishability accumulates at a finite rate
determined by the measurement channel, noise, and statistics. Operational
resolution bounds follow from rate-limited inference.}
\label{fig:overview}
\end{figure}

\begin{figure}[!ht]
\centering
\includegraphics[width=0.88\linewidth]{../figures/fig2_master_inequality_cartoon.pdf}
\caption{Schematic master inequality: ideal information accumulation versus an
upper bound suppressed by noise and finite-rate observation.}
\label{fig:master}
\end{figure}

\begin{figure}[!ht]
\centering
\includegraphics[width=0.88\linewidth]{../figures/fig3_phi_scaling.pdf}
\caption{Poisson-limited diffusion localization: $\deltat \propto \Flux^{-1/3}$
as a self-consistent fixed-point construction (Class~0A; non-inference).
The Monte Carlo curve provides a numerical consistency check of the construction
and the fitting pipeline under the specified observation model.}
\label{fig:phi_scaling}
\end{figure}

\begin{figure}[!ht]
\centering
\includegraphics[width=0.88\linewidth]{../figures/fig4_ou_gamma_bound.pdf}
\caption{OU-process / continuous monitoring bound: correlation-driven Fisher
information yields an operational limit on parameter resolution.}
\label{fig:ou}
\end{figure}

\begin{figure}[!ht]
\centering
\includegraphics[width=0.88\linewidth]{../figures/fig5_ramsey_phase_diagram.pdf}
\caption{Ramsey meeting-point (MP): inference-limited phase scale versus a
dynamical scale; the crossing indicates an operational boundary rather than a
scaling law.}
\label{fig:ramsey}
\end{figure}

\begin{figure}[!ht]
\centering
\includegraphics[width=0.88\linewidth]{../figures/fig6_mzi_visibility.pdf}
\caption{Mach--Zehnder meeting-point (MP): inference-limited phase scale versus a
geometric scale under visibility and noise constraints.}
\label{fig:mzi}
\end{figure}

\begin{figure}[!ht]
\centering
\includegraphics[width=0.88\linewidth]{../figures/fig7_noise_suppression_bound.pdf}
\caption{Noise suppresses Fisher accumulation: ideal information growth is
replaced by a bounded envelope under exponential suppression.}
\label{fig:noise}
\end{figure}

\begin{figure}[!ht]
\centering
\includegraphics[width=0.88\linewidth]{../figures/fig8_phi_slope_hist.pdf}
\caption{Multi-seed slope statistics for the diffusion $\Flux$-scaling experiment.
The distribution is consistent with the fixed-point construction exponent
$-1/3$ (Class~0A; non-inference) and reflects numerical stability of the fitting
procedure rather than inference validation.}
\label{fig:phi_hist}
\end{figure}

\begin{figure}[!ht]
\centering
\includegraphics[width=0.88\linewidth]{../figures/fig9_ctrw_alpha_sweep.pdf}
\caption{CTRW $\alpha$-sweep (Class~0B; exponent-imposed generator; non-inference):
fitted slopes versus the expected construction exponent $-1/(2+\alpha)$ across
transport exponents. The sweep serves as a numerical consistency check of the
generator and analysis pipeline, not as inference evidence.}
\label{fig:ctrw_alpha}
\end{figure}

\begin{figure}[!ht]
\centering
\includegraphics[width=0.88\linewidth]{../figures/fig10_ramsey_optimal_time.pdf}
\caption{Ramsey optimal interrogation time under decoherence: information gain
versus loss yields a finite optimum, illustrating an operational trade-off rather
than a scaling law.}
\label{fig:ramsey_opt}
\end{figure}

% ============================================================
% End of paper/figures/figures.tex
% ============================================================
