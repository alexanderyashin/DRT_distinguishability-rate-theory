\documentclass[11pt,a4paper]{article}

% ============================================================
% preamble.tex
% ============================================================
% LaTeX preamble for the DRT paper.
%
% IMPORTANT:
%  - This file MUST NOT contain \documentclass or \begin{document}.
%  - main.tex is the only compilation entry point.
%
% Notes:
%  - The package physics.sty is not available in the current TeXLive
%    environment. We use a minimal, portable substitute instead.
% ============================================================

% ------------------------------------------------------------
% Encoding and fonts
% ------------------------------------------------------------
\usepackage[T1]{fontenc}
\usepackage[utf8]{inputenc}
\usepackage{lmodern}

% ------------------------------------------------------------
% Page layout
% ------------------------------------------------------------
\usepackage{geometry}
\geometry{
  left=2.5cm,
  right=2.5cm,
  top=2.8cm,
  bottom=3.0cm
}

% ------------------------------------------------------------
% Typography and spacing
% ------------------------------------------------------------
\usepackage{microtype}
\usepackage{setspace}
\onehalfspacing

\setlength{\parindent}{0pt}
\setlength{\parskip}{0.6em}

% Overfull line mitigation (global, minimal)
\setlength{\emergencystretch}{2em}
\hfuzz=0.5pt

% ------------------------------------------------------------
% Mathematics
% ------------------------------------------------------------
\usepackage{amsmath}
\usepackage{amssymb}
\usepackage{amsthm}
\usepackage{bm}
\usepackage{mathtools}

% Theorem environments (used sparingly, operational focus)
\newtheorem{theorem}{Theorem}
\newtheorem{lemma}{Lemma}
\newtheorem{proposition}{Proposition}
\newtheorem{corollary}{Corollary}

% ------------------------------------------------------------
% Minimal "physics" substitutes (portable)
% ------------------------------------------------------------
\usepackage{braket}

% Derivatives
\newcommand{\dv}[2]{\frac{d #1}{d #2}}
\newcommand{\ddv}[2]{\frac{d^2 #1}{d #2^2}}
\newcommand{\pdv}[2]{\frac{\partial #1}{\partial #2}}
\newcommand{\pddv}[2]{\frac{\partial^2 #1}{\partial #2^2}}

% Absolute value and norm
\newcommand{\abs}[1]{\left\lvert #1 \right\rvert}
\newcommand{\norm}[1]{\left\lVert #1 \right\rVert}

% Expectation value (classical)
\newcommand{\expval}[1]{\left\langle #1 \right\rangle}

% ------------------------------------------------------------
% Figures and tables
% ------------------------------------------------------------
\usepackage{graphicx}
\usepackage{subcaption}
\usepackage{booktabs}
\usepackage{longtable}

% External PDF inclusion (figures generated from Evidence Pack)
\usepackage{pdfpages}

% ------------------------------------------------------------
% References and hyperlinks
% ------------------------------------------------------------
\usepackage{hyperref}
\hypersetup{
  colorlinks=true,
  linkcolor=blue,
  citecolor=blue,
  urlcolor=blue,
  pdfauthor={Alexander Yashin},
  pdftitle={Distinguishability as a Physical Primitive}
}

% ------------------------------------------------------------
% Lists
% ------------------------------------------------------------
\usepackage{enumitem}
\setlist{noitemsep, topsep=0.3em}

% ------------------------------------------------------------
% Utilities
% ------------------------------------------------------------
\usepackage{xcolor}

% ------------------------------------------------------------
% Custom commands (minimal, semantic only)
% ------------------------------------------------------------

% Distinguishability rate
\newcommand{\DistRate}{v_{\mathrm{D}}}

% Fisher Information
\newcommand{\FI}{\mathcal{I}}

% Quantum Fisher Information
\newcommand{\QFI}{\mathcal{F}_{Q}}

% Observation flux (photon / event rate)
\newcommand{\Flux}{\Phi}

% Minimal resolvable time
\newcommand{\deltat}{\delta t_{\min}}

% ------------------------------------------------------------
% Document hygiene
% ------------------------------------------------------------
\sloppy
\raggedbottom

% ============================================================
% End of preamble.tex
% ============================================================


\title{Distinguishability Rate Theory (DRT):\\ Operational limits of time and phase resolution from information-theoretic constraints}
\author{Alex Yashin}
\date{\today}

\begin{document}
\maketitle

\begin{abstract}
We present \emph{Distinguishability Rate Theory} (DRT), an operational framework that bounds time and phase resolution by the rate at which distinguishability (Fisher information / statistical separability) can be accumulated under realistic measurement constraints. The formalism yields universal scaling laws, including the self-consistent localization bound $\delta t_{\min}\propto \Phi^{-1/3}$ under Poisson-limited observation for normal diffusion, and $\delta t_{\min}\propto \Phi^{-1/(2+\alpha)}$ for anomalous diffusion with $\mathrm{MSD}\sim t^\alpha$. We connect these rate-limited inference bounds to standard interferometric settings (Ramsey and Mach--Zehnder), and provide numerically stable Monte Carlo simulations and reproducible figure-generation scripts.
\end{abstract}

\section{Overview}
DRT is motivated by a simple principle: \emph{inference is rate-limited}. Even if a parameter (time, phase, frequency) is physically encoded in a system, one can only resolve it as fast as information about that parameter is acquired through a measurement channel. The relevant quantity is an information accumulation functional (classical/quantum Fisher information), suppressed by noise and constrained by measurement statistics.

Figure~\ref{fig:overview} summarizes the conceptual pipeline used throughout this work.

\begin{figure}[ht]
  \centering
  \includegraphics[width=0.98\linewidth]{fig1_overview.pdf}
  \caption{DRT concept overview: distinguishability accumulates at a finite rate determined by the measurement channel, noise, and statistics. Operational resolution bounds follow from rate-limited inference.}
  \label{fig:overview}
\end{figure}

\section{Master inequality cartoon}
The central idea can be expressed as comparing an \emph{ideal} information accumulation to a \emph{noise-suppressed} upper bound. Figure~\ref{fig:master} provides the schematic structure used across scenarios.

\begin{figure}[ht]
  \centering
  \includegraphics[width=0.85\linewidth]{fig2_master_inequality_cartoon.pdf}
  \caption{Schematic master inequality: ideal information accumulation versus an upper bound suppressed by noise and finite-rate observation.}
  \label{fig:master}
\end{figure}

\section{Poisson-limited diffusion localization: $\Phi^{-1/3}$}
We consider one-dimensional Brownian motion with diffusion coefficient $D$, observed through Poisson-distributed photon detections with flux $\Phi$. Each detection yields a position estimate with Gaussian point-spread uncertainty $\sigma_m$. A self-consistent fixed-point argument yields the operational bound
\begin{equation}
  \delta t_{\min} \propto \Phi^{-1/3}.
\end{equation}
The Monte Carlo simulation confirms the scaling, producing a fitted slope close to $-1/3$ on log--log axes; see Figure~\ref{fig:phi}.

\begin{figure}[ht]
  \centering
  \includegraphics[width=0.85\linewidth]{fig3_phi_scaling.pdf}
  \caption{Monte Carlo confirmation of $\delta t_{\min}\propto \Phi^{-1/3}$ in Poisson-limited diffusion localization.}
  \label{fig:phi}
\end{figure}

\section{Continuous monitoring / OU-process bound}
For continuous monitoring with an Ornstein--Uhlenbeck process parameter $\gamma$, Fisher information accumulates only through temporal correlations; this yields an operational scaling for $\delta\gamma_{\min}$ that improves with observation time but is limited by noise. Figure~\ref{fig:ou} summarizes the bound used in the paper.

\begin{figure}[ht]
  \centering
  \includegraphics[width=0.85\linewidth]{fig4_ou_gamma_bound.pdf}
  \caption{OU-process / continuous monitoring bound: correlation-driven Fisher information yields an operational limit on parameter resolution.}
  \label{fig:ou}
\end{figure}

\section{Meeting-point phase diagrams: Ramsey and Mach--Zehnder}
DRT predicts a meeting-point where an inference-limited phase scale intersects a dynamical/geometric scale. We show the crossing in the Ramsey case (Figure~\ref{fig:ramsey}) and in the Mach--Zehnder case (Figure~\ref{fig:mzi}), illustrating operational equivalence under visibility/noise constraints.

\begin{figure}[ht]
  \centering
  \includegraphics[width=0.85\linewidth]{fig5_ramsey_phase_diagram.pdf}
  \caption{Ramsey meeting-point: inference-limited $\delta\phi_{\inf}$ versus dynamical scale $\delta\phi_{\mathrm{dyn}}$. The crossing indicates the operational boundary.}
  \label{fig:ramsey}
\end{figure}

\begin{figure}[ht]
  \centering
  \includegraphics[width=0.85\linewidth]{fig6_mzi_visibility.pdf}
  \caption{Mach--Zehnder meeting-point: inference-limited $\delta\phi_{\inf}$ versus geometric scale $\delta\phi_{\mathrm{geom}}$.}
  \label{fig:mzi}
\end{figure}

\section{Universal noise suppression of Fisher accumulation}
Noise reduces distinguishability rates. A generic exponential suppression factor $\exp(-2\Gamma t)$ yields an upper bound on accumulated Fisher information. Figure~\ref{fig:noise} visualizes the ideal accumulation versus the noise-suppressed envelope.

\begin{figure}[ht]
  \centering
  \includegraphics[width=0.85\linewidth]{fig7_noise_suppression_bound.pdf}
  \caption{Noise suppresses Fisher accumulation: an ideal linear growth is replaced by a bounded envelope under exponential suppression.}
  \label{fig:noise}
\end{figure}

\section{Reproducibility}
All results in this repository are reproducible via:
\begin{verbatim}
make doctor
make setup
make sims
make figs
make pdf
\end{verbatim}
Simulation outputs are stored in \texttt{results/*.json} and figures in \texttt{figures/*.pdf}.

\section*{Acknowledgements}
(Placeholder.)

\bibliographystyle{unsrt}
\bibliography{refs}

\end{document}
