% ============================================================
% sections/03_rate_limited_bounds.tex
% ============================================================

\section{Rate-Limited Bounds}
\label{sec:rate_limited_bounds}

This section establishes the formal core of Distinguishability-Rate Theory (DRT).
All results here are \emph{epistemic}: they follow solely from operational
distinguishability, fixed decision thresholds, and upper bounds on information
accumulation rates.
No assumptions about microscopic dynamics are made beyond what is encoded in the
observation channel and its distinguishability-rate envelope.

\subsection{Decision thresholds and operational resolution}

We recall the operational decision criterion introduced in
Section~\ref{sec:operational_framework}:
\begin{equation}
\delta\theta^{\mathsf T}\,\FI_T(\theta)\,\delta\theta
\;\ge\;
2\Dstar ,
\label{eq:rl_decision}
\end{equation}
where $\FI_T(\theta)$ denotes the (classical or quantum) Fisher information
accumulated over an observation window $T$, and $\Dstar>0$ is a fixed decision
threshold.
For a given $\FI_T(\theta)$, this criterion defines the minimal operationally
resolvable parameter scale $\delta\theta_{\min}$.

Rather than assuming $\FI_T(\theta)$ as given and bounding estimation error, DRT
asks which values of $\FI_T(\theta)$ are \emph{operationally achievable} under
finite-rate observation.

\subsection{Distinguishability-rate representation}

For any observation channel admitting local quadratic distinguishability, the
accumulated Fisher information can be written as
\begin{equation}
\FI_T(\theta)
\;=\;
\int_0^T \DistRate(t;\theta)\,dt ,
\end{equation}
where $\DistRate(t;\theta)$ is the instantaneous distinguishability rate.
No assumption of independent or identically distributed sampling is required.
All subsequent bounds arise from constraints on $\DistRate(t;\theta)$.

\subsection{Master inequality}

\begin{proposition}[Master inequality for rate-limited inference]
\label{prop:master_inequality}
Let $\theta$ be a parameter inferred through an observation channel with
accumulated Fisher or quantum Fisher information $\FI_T(\theta)$.
Assume that the distinguishability rate obeys an envelope
\begin{equation}
\DistRate(t;\theta)
\;\le\;
\DistRateMax(t)
\qquad
\text{for all } t\in[0,T],
\end{equation}
where $\DistRateMax(t)$ is fixed by the observation channel.
Then the minimal operationally resolvable scale along any direction $u$ satisfies
\begin{equation}
\delta\theta_{\min}(T)
\;\ge\;
\sqrt{
\frac{2\Dstar}{
\displaystyle \int_0^T \DistRateMax(t)\,dt
}
}.
\label{eq:rate_limited_bound}
\end{equation}
\end{proposition}

\paragraph{Assumptions.}
\begin{itemize}
\item[(A1)] Local (quadratic) distinguishability geometry holds in the relevant
regime.
\item[(A2)] A fixed decision threshold $\Dstar$ is chosen.
\item[(A3)] The observation channel admits an empirically characterizable upper
envelope $\DistRateMax(t)$ on distinguishability accumulation.
\end{itemize}

\paragraph{Proof (direct).}
By definition,
$\FI_T(\theta)=\int_0^T \DistRate(t;\theta)\,dt
\le \int_0^T \DistRateMax(t)\,dt$.
Insertion into the decision criterion~\eqref{eq:rl_decision} yields
Eq.~\eqref{eq:rate_limited_bound}.
\hfill$\square$

\paragraph{Interpretation.}
Proposition~\ref{prop:master_inequality} bounds not estimation error at fixed
information, but the \emph{achievable accumulation} of Fisher information itself
under finite-rate observation.
This inversion of the usual estimation-theoretic logic is central to DRT and
should not be confused with a reformulation of the Cramér--Rao bound.

\paragraph{Falsifiability.}
The proposition is falsified by demonstrating sustained accumulation of Fisher or
quantum Fisher information exceeding the envelope $\DistRateMax(t)$ within the
declared observation channel.

\subsection{Noise-suppressed accumulation}

In many experimental settings, distinguishability rates are suppressed by noise or
decoherence.
A generic envelope is
\begin{equation}
\DistRate(t;\theta)
\;\le\;
\DistRateIdeal(t)\,e^{-2\Gamma t},
\label{eq:noise_envelope}
\end{equation}
where $\Gamma$ is an effective noise or decoherence rate.
Integration yields
\begin{equation}
\FI_T(\theta)
\;\le\;
\int_0^T
\DistRateIdeal(t)\,e^{-2\Gamma t}\,dt .
\end{equation}

\begin{proposition}[Noise-limited inference saturation]
\label{prop:noise_limited}
If $\DistRateIdeal(t)$ is integrable on $[0,\infty)$, then
$\FI_T(\theta)$ converges to a finite value as $T\to\infty$.
Consequently, $\delta\theta_{\min}$ admits a nonzero lower bound even for
arbitrarily long observation times.
\end{proposition}

\paragraph{Interpretation.}
This saturation is epistemic.
It reflects exhaustion of information extractable through a noisy observation
channel, not intrinsic discreteness or stochasticity of the parameter $\theta$.

\subsection{Universality and dynamics-independence}

A defining feature of DRT is that operational resolution bounds depend only on
coarse, operationally accessible features of the observation channel.

\begin{lemma}[Dynamics-independence]
\label{lem:dynamics_independence}
Within a fixed observation channel class, two systems with distinct microscopic
dynamics but identical distinguishability-rate envelopes
$\DistRateMax(t)$ yield identical operational resolution bounds under DRT.
\end{lemma}

\paragraph{Operational content.}
Lemma~\ref{lem:dynamics_independence} is not a statement about microscopic
equivalence.
Its nontrivial content is that the envelope $\DistRateMax(t)$ is an
\emph{operationally characterizable quantity}, extractable from observable
statistics without access to underlying dynamics.
All microscopic details influence DRT bounds only insofar as they affect this
envelope.

\subsection{Regime Map and Validity Domains}
\label{sec:regime_map}

The bounds derived above apply within clearly defined operational regimes:

\begin{itemize}
\item \textbf{i.i.d.\ versus non-i.i.d.}
The master inequality holds without independence assumptions.
However, in non-i.i.d.\ continuous monitoring, Fisher information may reside
primarily in temporal correlations rather than in marginal statistics
(Section~\ref{sec:continuous_monitoring}).

\item \textbf{Asymptotic versus finite-resource regimes.}
Scaling laws derived from Eq.~\eqref{eq:rate_limited_bound} may fail at finite
$\Flux$, $N$, or $T$.
Such deviations are diagnostic of pre-asymptotic inference, not violations of the
bound.

\item \textbf{Noise- and decoherence-limited saturation.}
When distinguishability rates are exponentially suppressed, information
accumulation saturates and operational resolution ceases to improve with time
(Proposition~\ref{prop:noise_limited}).

\item \textbf{Classical, quantum-reducible, and ontological-residual regimes.}
All bounds in this section are reducible to Fisher or quantum Fisher geometry.
Resolution limits that persist after all reasonable rate constraints are relaxed
within a fixed channel class indicate the presence of non-epistemic structure and
are treated separately in
Section~\ref{sec:ontological_residues}.
\end{itemize}

\paragraph{Falsifiability hooks.}
For each regime, falsification requires controlled modification of the observation
channel or resource budget.
Observation of distinguishability accumulation exceeding the declared rate
envelopes would invalidate the corresponding DRT description.

\subsection{Inference-limited versus dynamics-limited regimes}

The master inequality~\eqref{eq:rate_limited_bound} defines two operational
regimes:
\begin{itemize}
  \item \textbf{Inference-limited regimes,} in which
  $\int_0^T \DistRateMax(t)\,dt$ grows too slowly for intrinsic dynamical scales to
  be resolved.
  \item \textbf{Dynamics-limited regimes,} in which information accumulation is
  sufficiently rapid that intrinsic geometric or dynamical scales dominate.
\end{itemize}
Meeting-point diagrams in
Sections~\ref{sec:classical_diffusion} and
\ref{sec:quantum_systems} visualize the crossover.

\subsection{Scope and boundary of applicability}

All bounds derived in this section are Fisher or quantum Fisher reducible and are
therefore epistemic in the sense of
Section~\ref{sec:definitions_criterion}.
If a resolution limit persists after all reasonable distinguishability-rate
constraints are relaxed within a fixed channel class, additional global or
algebraic structure is required.
Such cases are treated explicitly in
Section~\ref{sec:ontological_residues}.

% ============================================================
% End of sections/03_rate_limited_bounds.tex
% ============================================================
