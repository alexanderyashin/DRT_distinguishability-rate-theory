% ============================================================
% sections/00_abstract.tex  (FULL REPLACEMENT)
% ============================================================

\section*{Extended Abstract and Executive Framing}

This paper introduces \emph{Distinguishability Rate Theory} (DRT), an operational
framework for analyzing limits of resolution in time, phase, and related
parameters.
The framework is based on a single primitive: the finite rate at which
distinguishability can be accumulated through a specified observation channel.
Resolution bounds are derived from explicit information-theoretic constraints on
Fisher and quantum Fisher information under finite statistics, noise,
correlations, and measurement dynamics, without postulating ontological limits
\emph{a priori}.

\paragraph{Core principle.}
Resolution limits are treated as consequences of \emph{rate-limited inference}.
Whenever distinguishability accumulates at a bounded rate, a minimal resolvable
scale follows as an operational necessity.
As distinguishability rate, available statistics, or channel quality increase,
these limits relax.
Within this framework, a broad class of widely cited bounds is shown to be
\emph{epistemic} rather than ontological.

\paragraph{Main results.}
\begin{itemize}
\item \textbf{Universal rate-limited bound.}
A master inequality constrains achievable resolution by the envelope of
distinguishability accumulation imposed by a given observation channel
(Section~\ref{sec:rate_limited_bounds}).

\item \textbf{Self-consistent scaling laws.}
For Poisson-limited diffusion, the operational localization bound
$\delta t_{\min}\propto \Flux^{-1/3}$ is derived from a fixed-point inference
argument and confirmed numerically.
The same logic yields the generalized scaling
$\delta t_{\min}\propto \Flux^{-1/(2+\alpha)}$ for anomalous transport with
$\mathrm{MSD}\sim t^\alpha$
(Section~\ref{sec:classical_diffusion}).

\item \textbf{Non-\iid\ inference and correlation no-go.}
Continuous monitoring is shown to constitute an intrinsically non-\iid\
inference regime.
For stationary Ornstein--Uhlenbeck processes, Fisher information about parameters
resides exclusively in temporal correlations rather than marginals, yielding an
operational no-go for marginal estimation
(Section~\ref{sec:continuous_monitoring}).

\item \textbf{Reclassification of quantum limits.}
Many quantum interferometric bounds are reducible to constraints on quantum
Fisher information and therefore limit inference rather than encode ontological
structure.
Constraints that are not Fisher-reducible—such as those arising from
non-classical event structure—are isolated as genuine ontological residues
(Sections~\ref{sec:quantum_systems}–\ref{sec:ontological_residues}).
\end{itemize}

\paragraph{Epistemic--ontological boundary.}
DRT establishes a necessary and sufficient criterion for separating
inference-limited (epistemic) bounds from ontological constraints.
Epistemic bounds must relax under epistemic exhaustion, i.e.\ under unbounded
increase of distinguishability resources.
Constraints that persist under such exhaustion signal non-epistemic structure.

\paragraph{Evidence and falsifiability.}
All analytic results are supported by frozen, fully reproducible Monte Carlo
evidence.
Core claims are falsifiable by observing systematic violations of the predicted
rate-dependent scalings or by extracting parameter information from statistical
marginals in regimes where DRT predicts correlation-only access.

\paragraph{Scope.}
DRT provides a unifying operational language across classical and quantum systems
and a principled diagnostic for interpreting claimed fundamental limits.
It does not assert that all quantum constraints are epistemic; rather, it
identifies precisely which constraints are inference-limited and which require
independent ontological assumptions.

% ============================================================
% End of sections/00_abstract.tex
% ============================================================
