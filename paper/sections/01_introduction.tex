% ============================================================
% sections/01_introduction.tex  (FULL REPLACEMENT)
% ============================================================

\section{Introduction}
\label{sec:introduction}

\subsection*{Scope and stance}

This paper addresses a precise class of questions: \emph{how rapidly a physical
parameter can be operationally distinguished from data} when observation occurs
through a specified channel with finite statistics, finite bandwidth, and noise.
Throughout, we draw a strict boundary between
(i) \emph{epistemic limits} arising from rate-limited information accumulation and
(ii) \emph{ontological constraints} that cannot be reduced to local inference
geometry. Distinguishability-Rate Theory (DRT) formalizes this separation by
treating inference itself as a physical process subject to rate bounds.

\subsection*{Motivation: limits as rate constraints}

Parameters such as time $t$, phase $\phi$, or a relaxation rate $\gamma$ may be
well-defined within a model, yet experimentally accessible only via observation
channels with finite counts, finite flux $\Flux$, and noise. In such settings, the
relevant constraint is not the existence of the parameter, but the maximal rate at
which distinguishability can be accumulated.
DRT therefore replaces vague statements of ``fundamental limits'' by an explicit,
operational question:
\begin{quote}
\emph{Given a decision threshold $\Dstar$ and an observation channel with
finite-rate statistics, what is the minimal parameter variation that can be
distinguished within available resources $T$, $N$, or $\Flux$?}
\end{quote}
The organizing principle of the paper is a master inequality relating achievable
distinguishability to resource budgets and channel-induced information rates
(Figures~\ref{fig:overview} and~\ref{fig:master}).

\subsection*{Contributions}

The contributions of this paper are the following. Each is formulated under
explicit assumptions and is falsifiable within its stated regime.
\begin{itemize}
\item \textbf{Universal rate-limited bound (Proposition).}
A master inequality is derived that upper-bounds the accumulation of Fisher or
quantum Fisher information under finite-rate observation and noise
(Section~\ref{sec:rate_limited_bounds}).

\item \textbf{Self-consistent scaling constructions (Class 0A/0B; non-inference).}
For Poisson-limited localization of normal diffusion, a self-consistent
fixed-point closure yields the cubic-root construction
$\delta t_{\min} \propto \Flux^{-1/3}$ (Class~0A; not inference-supported).
For anomalous transport with $\mathrm{MSD}\sim t^\alpha$, the repository also
includes the generator-consistent construction
$\delta t_{\min} \propto \Flux^{-1/(2+\alpha)}$ (Class~0B; exponent-imposed
generator; non-inference)
(Section~\ref{sec:classical_diffusion};
Figures~\ref{fig:phi_scaling},~\ref{fig:ctrw_alpha}).

\item \textbf{Non-iid correlation no-go (Lemma/Theorem).}
In continuous monitoring, data are intrinsically non-iid.
For stationary Ornstein--Uhlenbeck processes, information about parameters resides
in temporal correlations rather than in marginal statistics, yielding a formal
no-go for marginal-only estimation
(Section~\ref{sec:continuous_monitoring}; Figure~\ref{fig:ou}).

\item \textbf{Reclassification of quantum limits (Propositions and No-Go
Statements).}
Quantum interferometric ``limits'' reducible to quantum Fisher information are
identified as epistemic.
Constraints not reducible to Fisher geometry are isolated as ontological residues
(Sections~\ref{sec:quantum_systems}–\ref{sec:ontological_residues};
Figures~\ref{fig:ramsey}–\ref{fig:mzi}).
\end{itemize}

\subsection*{Epistemic versus ontological limits}

DRT is intentionally restricted to constraints derivable from operational
distinguishability geometry and its accumulation rate. We adopt the following
technical classification, which is used consistently throughout the paper:
\begin{itemize}
  \item \textbf{Epistemic (inference-limited).}
  Bounds reducible to Fisher or quantum Fisher information accumulation under
  stated channel assumptions. Such bounds \emph{must} relax when the
  distinguishability rate or available resources increase.

  \item \textbf{Ontological residues.}
  Constraints that cannot be reduced to local Fisher or quantum Fisher geometry
  without invoking additional global, algebraic, or compatibility assumptions.
  These are treated explicitly in
  Section~\ref{sec:ontological_residues}.
\end{itemize}
Operationally, a bound that fails to relax under controlled increases of
distinguishability rate signals non-epistemic structure.

\subsection*{Relation to existing approaches}

Standard treatments of time--energy, phase, or metrological bounds often present
inequalities as fundamental without explicitly separating inference geometry from
ontological claims. DRT does not propose tighter constants.
Instead, it provides a unified rate-based mechanism explaining when such bounds
arise, how they scale with resources, and under which protocol changes they must
fail. Within DRT, ``limits'' function as diagnostic indicators of inference
structure rather than as axioms of nature.

\subsection*{What this paper does not claim}

For clarity, we explicitly state what is \emph{not} claimed:
\begin{itemize}
\item No claim is made that time, phase, or other parameters are ontologically
undefined.
\item No claim is made that all quantum bounds are epistemic; non-reducible
constraints are preserved as ontological.
\item No reinterpretation of quantum mechanics or replacement of its formalism is
proposed.
\end{itemize}

\subsection*{Regimes and falsifiability}

DRT distinguishes between:
(i) i.i.d.\ sampling and non-iid continuous monitoring;
(ii) finite-resource regimes where asymptotic scaling is not attained; and
(iii) noise-dominated regimes where distinguishability rates are exponentially
suppressed.
Predicted scalings and breakdowns across these regimes are mapped explicitly in
Section~\ref{sec:implications_falsifiability}.

The framework is falsifiable: sustained accumulation of Fisher or quantum Fisher
information exceeding the derived rate bounds—without modification of the
observation channel or resource budget—would falsify the rate-limited hypothesis
in the corresponding regime.

\subsection*{Reproducibility}

All numerical results and figures derive from a frozen Evidence Pack and are
reproducible via a one-command workflow (Appendix~E). Each quantitative claim is
traceable to deterministic scripts and stored outputs.

\subsection*{Paper organization}

Section~\ref{sec:operational_framework} introduces operational
distinguishability and decision thresholds.
Section~\ref{sec:rate_limited_bounds} derives the general rate-limited bounds.
Classical realizations are developed in
Sections~\ref{sec:classical_diffusion} and
\ref{sec:continuous_monitoring}.
Quantum systems are treated in Section~\ref{sec:quantum_systems}, and the
epistemic--ontological boundary is formalized in
Section~\ref{sec:ontological_residues}.
Regime diagnostics and falsifiability are collected in
Section~\ref{sec:implications_falsifiability}, followed by conclusions in
Section~\ref{sec:conclusions}.

% ============================================================
% End of sections/01_introduction.tex
% ============================================================
