% ============================================================
% sections/02_operational_framework.tex  (FULL REPLACEMENT)
% ============================================================

\section{Operational Framework}
\label{sec:operational_framework}

This section establishes the operational language used throughout the paper.
All notions introduced here are explicitly \emph{epistemic}: they concern
inference under finite data, finite-rate observation, and noise, and make no
ontological claims about the existence or nature of the underlying parameters.

\subsection{Observation channels and parameters}

\begin{definition}[Observation channel]
An \emph{observation channel} is a specification of how a parameter $\theta$
(e.g.\ time $t$, phase $\phi$, frequency $\omega$) encoded in a system gives rise
to observable data $Y_T$ over an observation window $T$, together with the
associated statistical model $p(Y_T|\theta)$.
\end{definition}

An observation channel may represent i.i.d.\ sampling, continuous monitoring, or
a quantum measurement protocol. Independence of data is \emph{not} assumed unless
stated explicitly. All bounds derived in this paper are conditional on the chosen
channel specification.

\subsection{Distinguishability as an operational primitive}

\begin{definition}[Local distinguishability]
Local distinguishability refers to the operational ability to discriminate
between two nearby hypotheses $\theta$ and $\theta+\delta\theta$ on the basis of
the observed data $Y_T$ generated by a fixed observation channel.
\end{definition}

For a classical observation channel with likelihood $p(Y_T|\theta)$, local
distinguishability is quantified by the accumulated Fisher information
\begin{equation}
\FI_T(\theta)
\;=\;
\mathbb{E}\!\left[
\left(\partial_\theta \log p(Y_T|\theta)\right)^2
\right].
\end{equation}
For quantum systems, $\FI_T(\theta)$ is replaced by the quantum Fisher
information $\mathcal{F}_{Q,T}(\theta)$, defined as the supremum of classical Fisher
information over all admissible measurements.

Both $\FI_T$ and $\mathcal{F}_{Q,T}$ are used here strictly as \emph{local
information metrics}: they define the quadratic geometry governing statistical
separability of nearby hypotheses. No assumption is made that the corresponding
Cramér--Rao bounds are saturable in finite time, under realistic noise, or under
arbitrary measurement constraints.

\subsection{Decision thresholds and operational resolution}

\begin{definition}[Decision threshold]
A \emph{decision threshold} $\Dstar$ is a dimensionless constant specifying the
minimal statistical separation required to reliably discriminate two hypotheses
under a fixed decision rule (e.g.\ bounded error probability or likelihood ratio
test).
\end{definition}

Local distinguishability is achieved when the quadratic form induced by the
accumulated information exceeds this threshold:
\begin{equation}
\delta\theta^{\mathsf T} \FI_T(\theta)\,\delta\theta
\;\ge\;
2\Dstar .
\label{eq:decision_threshold}
\end{equation}
This inequality defines an inference-limited minimal resolvable scale
\begin{equation}
\delta\theta_{\min}(T)
\;=\;
\sqrt{\frac{2\Dstar}{u^{\mathsf T} \FI_T(\theta)\,u}},
\end{equation}
where $u$ denotes the direction in parameter space under consideration.

\begin{remark}
Equation~\eqref{eq:decision_threshold} is purely operational and geometric in
nature. It expresses distinguishability in terms of a local quadratic form and
does not constitute a claim about optimal estimators, asymptotic efficiency, or
attainability.
\end{remark}

\subsection{Distinguishability rate}

\begin{definition}[Distinguishability rate]
Whenever differentiable, the \emph{distinguishability rate} is defined as
\begin{equation}
\DistRate(t;\theta)
\;=\;
\frac{d \FI_t(\theta)}{dt},
\end{equation}
with $\FI_t$ replaced by $\mathcal{F}_{Q,t}$ for quantum channels.
\end{definition}

In realistic measurement scenarios, $\DistRate$ is bounded by finite count rates,
finite bandwidth, decoherence, or other noise processes. The central operational
statement of DRT is that inference is limited by the \emph{integral} of
$\DistRate$, rather than by idealized microscopic dynamics alone.

This rate-based perspective separates two qualitatively distinct regimes:
\begin{itemize}
  \item \textbf{Dynamics-limited regimes,} in which information accumulates
  sufficiently rapidly that intrinsic system scales dominate achievable
  resolution.
  \item \textbf{Inference-limited regimes,} in which finite $\DistRate$ sets the
  dominant bound on resolution, independently of idealized dynamics.
\end{itemize}

\subsection{Accumulation structure and inference regimes}

The operational framework applies uniformly across different observation models:
\begin{itemize}
  \item \textbf{\iid\ sampling:} independent measurements with finite-count or
  finite-$\Flux$ statistics, for which $\FI_T$ is additive in time or number of
  samples.
  \item \textbf{\noniid\ continuous monitoring:} data streams in which
  information accumulates through temporal correlations and $\FI_T$ is generally
  non-additive (treated explicitly in
  Section~\ref{sec:continuous_monitoring}).
  \item \textbf{Quantum measurements:} channels constrained by quantum mechanics,
  for which $\mathcal{F}_{Q,T}$ provides an operational upper envelope on
  achievable classical Fisher information
  (Section~\ref{sec:quantum_systems}).
\end{itemize}
In all cases, the same decision inequality
\eqref{eq:decision_threshold} applies; only the structure and growth of the
accumulated information differ.

\subsection{Regime classification and scaling classes}

To prevent category errors and misinterpretation of scaling exponents, all
resolution results in DRT are classified by task type and epistemic class.
Scaling relations are valid \emph{only} within their originating class and must
not be transferred across rows of Table~\ref{tab:regime_classification}.

\begin{table}[t]
\centering
\caption{Classification of temporal-resolution regimes in
Distinguishability--Rate Theory.
Each scaling exponent is tied to a specific task class and must not be
reinterpreted outside that class.}
\label{tab:regime_classification}
\begin{tabular}{lllll}
\hline
\textbf{Task} &
\textbf{Observable} &
\textbf{Scaling} &
\textbf{Class} &
\textbf{Reference} \\
\hline
Inference-only estimation &
Local parameter offset $\delta\theta$ &
$\delta\theta_{\min}\propto \FI_T^{-1/2}$ &
Class~I &
App.~\ref{app:fisher_qfi} \\

Self-consistent temporal resolution &
Elapsed time $\delta t$ &
$\delta t_{\min}\propto \Phi^{-1/3}$ &
Class~0A &
App.~\ref{app:phi_minus_one_third} \\

Self-consistent anomalous transport &
Elapsed time $\delta t$ &
$\delta t_{\min}\propto \Phi^{-1/(2+\alpha)}$ &
Class~0A &
App.~\ref{app:ctrw_derivations} \\

Exponent-imposed simulation &
Synthetic estimator output &
Preset $\Phi^{-1/(2+\alpha)}$ &
Class~0B &
App.~\ref{app:ctrw_derivations} \\

Quantum inference (local) &
Parameter $\theta$ &
$\delta\theta_{\min}\propto \mathcal{I}_T^{-1/2}$ &
Class~I &
App.~\ref{app:fisher_qfi} \\
\hline
\end{tabular}
\end{table}

\subsection{Epistemic scope and non-claims}

The framework introduced here makes no ontological assertions. A bound on
$\delta\theta_{\min}$ derived from finite $\FI_T$ or $\mathcal{F}_{Q,T}$ states only
that, under the specified observation channel and decision threshold, finer
resolution is not operationally achievable. It does not imply that the parameter
itself is ill-defined, discrete, or fundamentally bounded.

Situations in which resolution limits cannot be reduced to local Fisher or quantum
Fisher geometry—without additional global, algebraic, or compatibility
assumptions—are identified later as \emph{ontological residues}
(Section~\ref{sec:ontological_residues}). Maintaining this separation is a core
methodological principle of DRT.

\subsection{Roadmap to rate-limited bounds}

With the operational language established, the next section derives general
rate-limited bounds by combining the decision inequality
\eqref{eq:decision_threshold} with explicit upper bounds on the distinguishability
rate $\DistRate$. These results form the technical core of DRT and underpin all
classical and quantum applications developed in the remainder of the paper.

% ============================================================
% End of sections/02_operational_framework.tex
% ============================================================
