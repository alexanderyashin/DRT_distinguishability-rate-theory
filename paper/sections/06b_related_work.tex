% ============================================================
% sections/06b_related_work.tex
% ============================================================

\section{Relation to Existing Approaches}
\label{sec:related_work}

This section positions Distinguishability-Rate Theory (DRT) with respect to
established approaches in estimation theory, quantum metrology, continuous
monitoring, and foundational discussions.
The purpose is not to provide a comprehensive literature review, but to clarify
scope, reuse, and non-claims.

\subsection{Classical estimation theory}

Classical estimation theory provides a rigorous framework for bounding estimation
error given an assumed statistical model.
Canonical results include the Cramér--Rao bound, its Bayesian extensions, and
local asymptotic normality (e.g.\ the work of Kay, Van Trees, and Le Cam).
These results characterize the achievable precision \emph{conditional on} a
given Fisher information.

DRT does not modify these bounds.
Instead, it addresses a complementary question not formulated in classical
estimation theory: which Fisher information values are operationally achievable
under finite-rate observation.
In this sense, DRT treats Fisher information not as an input, but as an object
constrained by the observation process itself.

\subsection{Quantum metrology}

Quantum metrology extends estimation theory to quantum states and measurements,
introducing the quantum Fisher information and bounds due to Helstrom and Holevo.
In idealized settings, these results identify ultimate precision limits under
optimal measurements.

A substantial body of work has shown that in realistic settings---including
noise, decoherence, and continuous monitoring---quantum Fisher information
accumulation is strongly suppressed and often saturates.
DRT does not contest these results.
Rather, it provides a unifying operational interpretation: such limits arise from
rate constraints on distinguishability accumulation and are therefore epistemic
within the declared observation channel.

Importantly, DRT does not claim new quantum limits.
It provides a diagnostic criterion for determining when a claimed quantum bound
reflects inference limitations rather than non-epistemic structure.

\subsection{Continuous monitoring and non-i.i.d.\ inference}

Many physical inference problems involve continuous-time measurements, correlated
noise, or non-i.i.d.\ data streams.
Filtering theory and stochastic process methods describe how information is
distributed across time and correlations in such settings.

DRT builds on these insights by treating distinguishability rate as a primitive
operational quantity.
This perspective clarifies why, for example, parameters in Ornstein--Uhlenbeck
processes are identifiable only through temporal correlations and why no amount
of snapshot sampling suffices.
The contribution of DRT is not a new filtering method, but an explicit rate-based
bound that unifies such observations across models.

\subsection{Foundational interpretations}

Operational bounds on time, phase, or parameter resolution are often interpreted
as statements about the fundamental structure of physical reality.
DRT explicitly refrains from such interpretations.

Within the framework introduced here, inference-limited bounds are epistemic:
they arise from local distinguishability geometry and finite information rates.
Persistent constraints that do not relax under epistemic exhaustion are treated
separately as non-epistemic residues within a fixed observation channel class.
DRT does not explain or derive such structures; it identifies the point at which
inference-based reasoning ceases to apply.

\subsection{Summary of scope}

In summary, DRT:
\begin{itemize}
\item reuses standard estimation-theoretic and quantum-metrological results,
\item reinterprets many resolution limits as consequences of finite
distinguishability rates,
\item introduces no new dynamical or ontological assumptions, and
\item provides an operational criterion for separating inference-limited bounds
from non-epistemic constraints.
\end{itemize}

Claims beyond this scope are intentionally excluded.

% ============================================================
% End of sections/06b_related_work.tex
% ============================================================
