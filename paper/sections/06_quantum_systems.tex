% ============================================================
% sections/06_quantum_systems.tex  (FULL REPLACEMENT)
% ============================================================

\section{Quantum Systems}
\label{sec:quantum_systems}

This section applies Distinguishability-Rate Theory (DRT) to quantum-mechanical
settings. The goal is twofold:
(i) to show that a broad class of quantum ``limits'' are operational consequences
of bounded distinguishability rates, and
(ii) to isolate precisely those constraints that are \emph{not} reducible to
local Fisher or quantum Fisher information (QFI) geometry and therefore require
additional non-epistemic structure.
Throughout, we enforce a strict classification between inference-limited
(QFI-reducible) bounds and non-Fisher-reducible constraints.

\subsection{Quantum Fisher information as an epistemic ceiling}

Let $\rho_\theta$ be a quantum statistical model encoding a parameter $\theta$.
The quantum Fisher information $\QFI(\theta)$ is defined as the supremum of the
classical Fisher information over all admissible measurements on $\rho_\theta$.
Operationally, $\QFI(\theta)$ quantifies the maximal \emph{local} (quadratic)
distinguishability permitted by quantum mechanics for a single use of the model.

To interface with DRT, we distinguish:
(i) the local geometric quantity $\QFI(\theta)$, and
(ii) the \emph{protocol- and channel-dependent} quantity $\FI_T^{Q}(\theta)$,
defined as the maximal Fisher information \emph{achievable over an observation
window $T$} under the specified measurement channel class (including sampling,
control, and noise constraints). In DRT language, $\FI_T^{Q}(\theta)$ is the
quantum analogue of an accumulated distinguishability budget.

\begin{proposition}[QFI as supremum over Fisher information]
\label{prop:qfi_supremum}
For any measurement strategy applied to $\rho_\theta$ within the declared channel
class, the accumulated classical Fisher information satisfies
\[
\FI_T(\theta) \;\le\; \FI_T^{Q}(\theta).
\]
Equality is achievable only for optimal measurements matched to the local
eigenstructure of the symmetric logarithmic derivative and compatible with the
channel constraints.
\end{proposition}

\paragraph{Assumptions.}
\begin{itemize}
\item[(Q1)] Valid quantum statistical model $\rho_\theta$;
\item[(Q2)] admissible measurements obey quantum measurement postulates and the
declared channel constraints;
\item[(Q3)] local (quadratic) distinguishability geometry in the regime of
interest.
\end{itemize}

\paragraph{Interpretation.}
Any resolution bound expressible solely in terms of $\FI_T^{Q}(\theta)$ constrains
\emph{inference} under a specified channel class. It is therefore epistemic and
does not, by itself, impose an ontological restriction on $\theta$.

Combining Proposition~\ref{prop:qfi_supremum} with the operational decision
criterion yields
\begin{equation}
\delta\theta_{\min}
\;\ge\;
\sqrt{\frac{2\Dstar}{\FI_T^{Q}(\theta)}} ,
\label{eq:qfi_operational_bound}
\end{equation}
formally identical to the classical bound with $\FI_T$ replaced by $\FI_T^{Q}$.

\subsection{Mandelstam--Tamm as a rate-limited inequality}

Consider unitary parameter encoding
$\rho_t = e^{-iHt}\rho_0 e^{iHt}$ with generator $H$.
Unless stated otherwise, we work in units $\hbar=1$ (the restoration
$H\!\to\! H/\hbar$ is immediate).

For the unitary family generated by $H$, the quantum Fisher information for the
time parameter satisfies
\[
\QFI(t) = 4\,\mathrm{Var}_{\rho_0}(H)\, t^2
\qquad
(\text{equivalently } \QFI(t)=4\,\mathrm{Var}_{\rho_0}(H)\, t^2/\hbar^2).
\]

\begin{proposition}[Mandelstam--Tamm as distinguishability-rate bound]
\label{prop:mt_as_rate}
The Mandelstam--Tamm bound is QFI-reducible: it follows from the maximal local
distinguishability rate set by the generator variance $\mathrm{Var}_{\rho_0}(H)$.
Consequently, it is epistemic in the DRT sense.
\end{proposition}

\paragraph{Assumptions.}
\begin{itemize}
\item[(M1)] Unitary parameter encoding generated by $H$;
\item[(M2)] finite generator variance $\mathrm{Var}_{\rho_0}(H)$;
\item[(M3)] absence of additional decoherence beyond the declared unitary model.
\end{itemize}

\paragraph{Derivation (one line).}
Insert $\FI_T^{Q}(t)=\QFI(t)$ into Eq.~\eqref{eq:qfi_operational_bound} to obtain
\[
\delta t_{\min} \ge \frac{\sqrt{2\Dstar}}{2\sqrt{\mathrm{Var}(H)}} ,
\qquad
\text{or (restoring $\hbar$): }
\delta t_{\min} \ge \frac{\hbar\sqrt{2\Dstar}}{2\sqrt{\mathrm{Var}(H)}} .
\]
This is the operational content of the Mandelstam--Tamm inequality in DRT form.
\hfill$\square$

\paragraph{Interpretation.}
Within DRT, Mandelstam--Tamm limits the \emph{rate} at which distinguishability
can accumulate under unitary evolution. It does \emph{not} assert a fundamental
quantum of time.

\subsection{Decoherence and meeting-point structure}

In realistic quantum systems, decoherence suppresses distinguishability rates.
Let ${\DistRate}^{Q}(t;\theta)$ denote the instantaneous quantum
distinguishability rate (the QFI-rate analogue of
Section~\ref{sec:rate_limited_bounds}).
A generic noise envelope takes the form
\begin{equation}
{\DistRate}^{Q}(t;\theta)
\;\le\;
{\DistRate}^{Q,\mathrm{ideal}}(t)\,e^{-2\Gamma t},
\label{eq:qfi_noise_envelope}
\end{equation}
where $\Gamma$ is an effective decoherence/noise rate.
Integrating yields a saturation-type bound on $\FI_T^{Q}(\theta)$ analogous to
Proposition~\ref{prop:noise_limited}.

As a result, there exists an optimal interrogation time beyond which additional
evolution degrades inference. This ``meeting-point'' between inference-limited
and dynamics-limited regimes is demonstrated for Ramsey interferometry
(Figure~\ref{fig:ramsey}) and Mach--Zehnder interferometry with finite visibility
(Figure~\ref{fig:mzi}). The frozen numerical data confirm the predicted behavior
without introducing additional assumptions.

\subsection{Classification of quantum bounds}

We now formalize the epistemic--ontological separation.

\begin{theorem}[QFI-reducible quantum bounds]
\label{thm:qfi_reducible}
Any resolution bound derivable solely from an upper bound on local quantum Fisher
information accumulation (equivalently, from an envelope on ${\DistRate}^{Q}$) is
epistemic. Such bounds constrain inference under a specified observation channel
and must relax if distinguishability rates are increased (e.g.\ by reducing noise,
increasing effective visibility, or modifying measurement protocols within the
channel class).
\end{theorem}

\paragraph{Interpretation.}
Theorem~\ref{thm:qfi_reducible} classifies a broad class of quantum metrological
``limits'' as inference-limited rather than ontological.

\begin{proposition}[Additional-structure-dependent bounds]
\label{prop:ml_status}
Bounds requiring assumptions beyond local QFI geometry---for example, global
spectral constraints or reference-energy assumptions for the generator $H$---are
not purely Fisher/QFI-reducible. Their validity depends explicitly on these
additional assumptions and therefore lies outside rate-limited inference alone.
\end{proposition}

\paragraph{Example.}
Margolus--Levitin-type bounds rely on global energy constraints (e.g.\ a
reference to a ground-state energy) and do not arise from local Fisher/QFI
geometry or distinguishability-rate limitations alone.

\begin{lemma}[Non-Fisher-reducible quantum constraints]
\label{lem:non_fisher_quantum}
Constraints originating from non-classical event algebra, contextuality, or
global compatibility conditions cannot be reduced to local Fisher or QFI geometry.
They do not relax under controlled increases of distinguishability rate within a
fixed local-inference framework and therefore represent non-epistemic structure.
\end{lemma}

\paragraph{Examples.}
Bell and Kochen--Specker constraints fall into this class. Their violation cannot
be generated or removed by modifying distinguishability rates within any local
Fisher/QFI-reducible inference framework.

\subsection{Noise as a universal suppressor}

Noise and decoherence universally suppress distinguishability rates in quantum
systems. The exponential envelopes in
Eq.~\eqref{eq:qfi_noise_envelope} mirror the classical noise-limited bounds of
Section~\ref{sec:rate_limited_bounds}, explaining the structural universality of
rate-limited inference across classical and quantum domains.

\subsection{Falsifiability}

The epistemic classification advanced in this section is experimentally
falsifiable. Sustained accumulation of Fisher or quantum Fisher information
exceeding the stated rate envelopes---without modification of the observation
channel or introduction of additional degrees of freedom---would refute the
rate-limited hypothesis for the corresponding quantum system.

\subsection{Summary}

Quantum systems do not invalidate DRT. They provide a setting in which the
boundary between inference-limited and non-epistemic constraints can be drawn
sharply. Local QFI-reducible bounds are epistemic; genuinely non-classical
constraints require additional global or algebraic structure. This prepares the
ground for the explicit analysis of ontological residues in the next section.

% ============================================================
% End of sections/06_quantum_systems.tex
% ============================================================
