% ============================================================
% sections/08_implications_falsifiability.tex  (FULL REPLACEMENT)
% ============================================================

\section{Implications and Falsifiability}
\label{sec:implications_falsifiability}

This section converts the epistemic--ontological separation into explicit,
testable scientific statements. We provide falsifiability criteria, an explicit
\emph{Regime Map and Validity Domains}, and diagnostic rules for interpreting
deviations without category errors.

\subsection{Core falsifiable propositions}

All claims below are conditional on a specified observation channel and stated
assumptions.

\begin{proposition}[Rate-limited scaling]
\label{prop:fals_rate}
For a fixed observation channel, operational resolution improves no faster than
the envelope determined by Fisher or quantum Fisher information accumulation.
\end{proposition}

\begin{proposition}[Universality across realizations]
\label{prop:fals_universal}
Distinct physical realizations sharing the same effective
distinguishability-rate envelope exhibit identical operational scaling laws,
independent of microscopic dynamics.
\end{proposition}

\begin{proposition}[Noise-induced saturation]
\label{prop:fals_noise}
If distinguishability rates are exponentially suppressed by noise or
decoherence, information accumulation saturates, producing a finite optimal
interrogation time (a meeting point).
\end{proposition}

\paragraph{Falsification rule.}
Persistent and reproducible violation of any proposition above—without modifying
the observation channel or introducing additional independent resources—falsifies
DRT for the corresponding regime.

\subsection{Regime Map and Validity Domains}

DRT partitions inference problems into diagnostically distinct regimes. Each
regime includes its validity domain and falsifiability hook.

\begin{itemize}
\item \textbf{i.i.d.\ inference regime.}
\emph{Domain:} Independent detection events with finite count or flux statistics.
\emph{Signature:} Poissonian accumulation; power-law scalings (e.g.\
$\Flux^{-1/3}$).
\emph{Falsifiability:} Super-Poissonian accumulation without new channels.

\item \textbf{Correlation-limited (non-\iid) regime.}
\emph{Domain:} Continuous monitoring; correlated data streams.
\emph{Signature:} Information in temporal correlations; linear-in-$T$ asymptotics.
\emph{Falsifiability:} Super-linear accumulation without altered correlations.

\item \textbf{Noise/decoherence-limited regime.}
\emph{Domain:} Exponential suppression of distinguishability rates.
\emph{Signature:} Saturation and meeting-point behavior.
\emph{Falsifiability:} Unbounded growth without reducing noise or changing
protocols.

\item \textbf{Finite-resource (pre-asymptotic) regime.}
\emph{Domain:} Short $T$ or low $\Flux$.
\emph{Signature:} Transient deviations from asymptotic scaling.
\emph{Falsifiability:} Persistence of deviations after increasing resources.

\item \textbf{Ontological regime.}
\emph{Domain:} Constraints persisting under epistemic exhaustion.
\emph{Signature:} Non-relaxable bounds not Fisher/QFI-reducible.
\emph{Falsifiability:} Relaxation under increased distinguishability would
invalidate ontological classification.
\end{itemize}

\subsection{Diagnostic interpretation of deviations}

Observed deviations are interpreted as follows:
\begin{itemize}
\item \textbf{Steeper-than-predicted scaling} $\Rightarrow$ hidden independent
channels or additional degrees of freedom.
\item \textbf{Weaker scaling or early saturation} $\Rightarrow$ noise,
decoherence, or model misspecification.
\item \textbf{Non-relaxable bounds} $\Rightarrow$ ontological residues rather than
inference limits.
\end{itemize}

\subsection{Experimental programs}

\paragraph{Classical diffusion.}
Vary photon flux to test $\Flux^{-1/3}$ and $\Flux^{-1/(2+\alpha)}$ scalings and
collapse across systems with different $D$.

\paragraph{Continuous monitoring.}
Long-duration monitoring of correlated processes to test linear-in-$T$
accumulation and exclude super-linear growth without new channels.

\paragraph{Quantum interferometry.}
Controlled tuning of visibility and decoherence in Ramsey and Mach--Zehnder
interferometry to observe meeting points and saturation.

\subsection{Explicit non-claims}

For clarity:
\begin{itemize}
\item No claim of fundamental discreteness of time, phase, or parameters.
\item No prohibition of improved resolution via genuinely new resources.
\item No interpretation of quantum mechanics is proposed.
\end{itemize}

\subsection{Broader implications}

DRT reframes ``limits'' as diagnostics of inference architecture.
Epistemic bounds acquire precise operational meaning, while genuinely
non-classical features are isolated as ontological structure.

% ============================================================
% End of sections/08_implications_falsifiability.tex
% ============================================================
