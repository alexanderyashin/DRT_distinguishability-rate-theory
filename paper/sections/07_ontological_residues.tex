% ============================================================
% sections/07_ontological_residues.tex  (FULL REPLACEMENT)
% ============================================================

\section{Non-Epistemic Residues within Fixed Channel Classes}
\label{sec:ontological_residues}

The preceding sections derived a hierarchy of rate-limited bounds arising from
local distinguishability geometry and finite information accumulation.
Using the operational criterion introduced in
Section~\ref{sec:definitions_criterion}, we now analyze which constraints
\emph{persist} once all Fisher-reducible limitations within a fixed observation
channel class have been exhausted.

Throughout this section, ``non-epistemic'' is used in a \emph{strictly operational
sense}: it denotes constraints that cannot be removed by improving inference
within the declared channel class.
No claim is made about ultimate ontology beyond that scope.

\subsection{Non-epistemic residues}

\begin{definition}[Non-epistemic residue]
Within a fixed observation channel class $\mathcal{C}$, a
\emph{non-epistemic residue} is a constraint that:
\begin{itemize}
\item[(i)] persists under epistemic exhaustion within $\mathcal{C}$, as defined in
Section~\ref{sec:definitions_criterion}, and
\item[(ii)] cannot be expressed as, or reduced to, a Fisher- or QFI-reducible bound
within $\mathcal{C}$.
\end{itemize}
\end{definition}

Operationally, a non-epistemic residue marks the point at which inference-based
improvements saturate without eliminating the constraint, given the fixed event
structure and admissible operations defining $\mathcal{C}$.

\subsection{Examples and non-examples}

\paragraph{Purely epistemic bounds.}
Mandelstam--Tamm-type speed limits, noise-suppressed Ramsey bounds, and
Poisson-limited localization bounds all relax under epistemic exhaustion when the
distinguishability rate envelope within their respective channel classes is
expanded.
They are therefore epistemic according to
Theorem~\ref{thm:operational_criterion}.

\paragraph{Intermediate cases.}
Bounds such as the Margolus--Levitin limit depend explicitly on additional global
assumptions (e.g.\ spectral references or energy constraints).
Their classification is conditional on these assumptions and therefore lies
outside the strict diagnostic scope of DRT.
DRT neither affirms nor rejects such bounds; it isolates the assumptions on which
they rest.

\paragraph{Persistent constraints.}
Constraints associated with non-classical event structure—such as Bell or
Kochen--Specker inequalities—persist under epistemic exhaustion within any channel
class whose admissible operations preserve classical event algebra.
Within that scope, they constitute non-epistemic residues.

\subsection{No-Free-Epistemic-Lunch principle}

\begin{theorem}[No-Free-Epistemic-Lunch (NFEL)]
\label{thm:nfel}
Within a fixed observation channel class $\mathcal{C}$, no sequence of epistemic
operations—i.e.\ operations acting solely on Fisher- or QFI-reducible structures
within $\mathcal{C}$—can generate, remove, or violate constraints arising from
non-classical event structure.
\end{theorem}

\paragraph{Assumptions.}
\begin{itemize}
\item[(N1)] Epistemic operations modify only local probability distributions,
density operators, or control parameters admissible within $\mathcal{C}$.
\item[(N2)] The event structure defining $\mathcal{C}$ (e.g.\ classical or
non-classical compatibility relations) is held fixed.
\end{itemize}

\paragraph{Proof sketch.}
Epistemic operations act within the local distinguishability geometry associated
with $\mathcal{C}$.
They can modify Fisher- or QFI-reducible quantities but do not alter the global
event structure or compatibility relations that define which joint events are
admissible.
Constraints rooted in this event structure therefore remain invariant under
epistemic exhaustion.
\hfill$\square$

\subsection{Interpretational role}

Non-epistemic residues delineate the boundary of applicability of
inference-based reasoning within a fixed channel class.
DRT does not elevate these residues to universal ontological statements.
Instead, it identifies them as constraints that cannot be removed by improving
distinguishability rates, decision rules, or estimation protocols alone, without
altering the underlying event structure.

\subsection{Relation to quantum theory}

In quantum mechanics, non-epistemic residues correspond to global compatibility,
contextuality, and non-Boolean event structure.
These features are invisible to local Fisher or QFI geometry and therefore lie
outside the scope of rate-limited inference.
DRT neither derives nor negates such structures; it specifies where
inference-based diagnostics cease to apply.

\subsection{Falsifiability boundary}

The classification introduced here is operationally falsifiable.
A purported non-epistemic residue would be reclassified as epistemic if a protocol
operating entirely within the declared channel class $\mathcal{C}$ were shown to
remove or violate it through improved inference alone.
Conversely, persistence of the constraint under exhaustive epistemic variation
confirms its status as a non-epistemic residue relative to $\mathcal{C}$.

\subsection{Transition}

Having isolated the boundary between inference-limited bounds and
non-epistemic residues, we now translate this classification into explicit regime
maps and falsifiability diagnostics.
The next section consolidates these results into experimentally testable criteria.

% ============================================================
% End of sections/07_ontological_residues.tex
% ============================================================
