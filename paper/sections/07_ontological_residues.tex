% ============================================================
% sections/07_ontological_residues.tex
% ============================================================

\section{Non-Epistemic Residues within Fixed Channel Classes}
\label{sec:ontological_residues}

The preceding sections established a variety of rate-limited bounds arising from
local distinguishability and inference.
Using the operational framework introduced in
Section~\ref{sec:definitions_criterion}, we now analyze which constraints persist
once all Fisher-reducible limitations within a fixed observation channel class
have been exhausted.

\subsection{Non-epistemic residues}

\begin{definition}[Non-epistemic residue]
Within a fixed observation channel class $\mathcal{C}$, a
\emph{non-epistemic residue} is a constraint that:
\begin{itemize}
\item[(i)] persists under epistemic exhaustion as defined in
Section~\ref{sec:definitions_criterion}, and
\item[(ii)] cannot be expressed as a Fisher-reducible bound within $\mathcal{C}$.
\end{itemize}
\end{definition}

Such residues indicate limitations not attributable to inference or local
distinguishability within the declared scope.

\subsection{Examples and non-examples}

\paragraph{Epistemic bounds.}
Mandelstam--Tamm--type speed limits, noise-suppressed Ramsey bounds, and
Poisson-limited localization vanish under epistemic exhaustion within their
respective channel classes.
They are therefore epistemic in the sense of
Theorem~\ref{thm:operational_criterion}.

\paragraph{Intermediate cases.}
Bounds such as the Margolus--Levitin limit depend on additional global assumptions
(e.g.\ spectral constraints).
Their classification is conditional on these assumptions and lies outside the
strict diagnostic scope of DRT.

\paragraph{Persistent constraints.}
Constraints associated with non-classical event structure---such as Bell or
Kochen--Specker inequalities---persist under epistemic exhaustion within any
channel class that preserves classical event algebra.
Within this scope, they constitute non-epistemic residues.

\subsection{No-Free-Epistemic-Lunch principle}

\begin{theorem}[No-Free-Epistemic-Lunch (NFEL)]
\label{thm:nfel}
Within a fixed observation channel class $\mathcal{C}$, no sequence of epistemic
operations can generate, remove, or violate constraints arising from non-classical
event structure.
\end{theorem}

\paragraph{Assumptions.}
\begin{itemize}
\item[(N1)] Epistemic operations act only on local probability distributions or
density operators within $\mathcal{C}$.
\item[(N2)] Event-structure constraints encode global compatibility relations not
captured by local distinguishability.
\end{itemize}

\paragraph{Proof sketch.}
Epistemic operations modify Fisher-reducible quantities while preserving the
event structure defining $\mathcal{C}$.
Constraints rooted in non-classical event structure therefore remain invariant
under epistemic exhaustion.
\hfill$\square$

\subsection{Interpretational role}

Non-epistemic residues mark the boundary of applicability of inference-based
reasoning.
DRT does not interpret these residues as fundamental ontological claims beyond
the declared channel class; it identifies them as constraints not removable by
improved inference alone.

\subsection{Relation to quantum theory}

In quantum mechanics, non-epistemic residues correspond to global compatibility
and contextuality constraints.
These structures are invisible to local information geometry and lie outside the
scope of rate-limited inference.
DRT neither derives nor negates them; it delineates where inference-based analysis
terminates.

\subsection{Transition to falsifiability}

Having operationally isolated non-epistemic residues, we now translate this
classification into experimentally testable diagnostics.
The next section formulates explicit regime maps and falsifiability criteria.

% ============================================================
% End of sections/07_ontological_residues.tex
% ============================================================
