% ============================================================
% sections/05_continuous_monitoring.tex  (FULL REPLACEMENT)
% ============================================================

\section{Continuous Monitoring}
\label{sec:continuous_monitoring}

Sections~\ref{sec:rate_limited_bounds} and
\ref{sec:classical_diffusion} addressed inference from discrete or effectively
\iid\ detection events.
Many experimentally relevant scenarios, however, involve
\emph{continuous monitoring}, where information about a parameter is encoded
primarily in temporal correlations rather than in independent samples.
This section formalizes such intrinsically \noniid\ regimes within DRT and
demonstrates that they constitute a qualitatively distinct inference class.

\subsection{Correlated inference as a distinct regime}

In continuous monitoring, the observation channel produces a stochastic time
series $Y(t)$ whose samples are correlated.
As a consequence, Fisher information about a parameter $\theta$ is generally
\emph{not} determined by marginal distributions alone.
It resides in joint statistics and, in particular, in temporal correlation
structure.

Operationally, this implies that distinguishability accumulates only through
access to correlations.
Treating correlated data streams as \iid\ samples overestimates achievable
resolution by implicitly assuming access to independent information channels
that are not present.
DRT therefore treats continuous monitoring as a distinct inference regime,
not as a limiting case of high-rate \iid\ sampling.

\subsection{Ornstein--Uhlenbeck dynamics as a canonical model}

A minimal analytically tractable example is the stationary
Ornstein--Uhlenbeck (OU) process,
\begin{equation}
dX_t = -\gamma X_t\,dt + \sqrt{2D}\,dW_t ,
\end{equation}
where $\gamma>0$ is the relaxation rate and $D$ sets the noise strength.
We consider continuous observation of $X_t$ over a time window $T$, with the goal
of inferring $\gamma$.

The OU process is not invoked as a special physical system, but as the simplest
stochastic process with a finite correlation time and analytically controlled
statistics.
The conclusions below rely only on these structural features.

\begin{lemma}[Marginal identifiability no-go for stationary OU]
\label{lem:ou_marginal_nogo}
Consider the stationary OU process with unknown diffusion strength $D$ treated as
a nuisance parameter.
If inference is restricted to \emph{single-time} (marginal) statistics of $X_t$
(i.e., unordered samples without temporal correlations), then $\gamma$ is not
identifiable:
the marginal distribution determines only the ratio $D/\gamma$ and cannot
operationally fix the correlation time $\gamma^{-1}$.
Equivalently, all operationally usable information about the \emph{rate} $\gamma$
resides in temporal correlations.
\end{lemma}

\paragraph{Assumptions.}
\begin{itemize}
\item[(O1)] Stationary OU dynamics;
\item[(O2)] inference uses only single-time marginals of $X_t$
(unordered samples; no access to temporal correlations);
\item[(O3)] $D$ is not assumed known a priori (nuisance parameter).
\end{itemize}

\paragraph{Proof sketch.}
In stationarity, $X_t$ is Gaussian with mean $0$ and variance
$\mathrm{Var}(X_t)=D/\gamma$.
The marginal distribution therefore depends on parameters only through the
combination $D/\gamma$.
Without independent information fixing $D$, the mapping
$(\gamma,D)\mapsto D/\gamma$ is many-to-one, so $\gamma$ is not identifiable from
single-time statistics alone.
By contrast, the autocorrelation function
$\langle X_t X_{t+\tau}\rangle = (D/\gamma)\,e^{-\gamma \tau}$ contains an explicit
timescale $\gamma^{-1}$, making $\gamma$ accessible only through temporal
correlations.
\hfill$\square$

\paragraph{Interpretation.}
Lemma~\ref{lem:ou_marginal_nogo} is a strict operational no-go statement.
Marginal statistics can fix a stationary \emph{scale} (here $D/\gamma$), but not
the \emph{rate} $\gamma$ governing correlation decay.
This is the precise sense in which, under continuous monitoring,
\emph{time information resides in correlations}.

\subsection{Correlation-limited information accumulation}

Because information resides in correlations, distinguishability accumulates only
on timescales exceeding the correlation time $\gamma^{-1}$.
For continuous monitoring of stationary OU dynamics, the accumulated Fisher
information about $\gamma$ grows linearly with $T$ at a rate fixed by the
correlation structure:
\begin{equation}
\FI_T(\gamma) \;\simeq\; \frac{T}{2\gamma} ,
\end{equation}
up to channel-dependent prefactors.
This scaling reflects the fact that effectively independent information units are
obtained only once per correlation time.

\begin{proposition}[OU correlation-limited bound]
\label{prop:ou_bound}
Under continuous monitoring of stationary OU dynamics satisfying
assumptions (O1)--(O3), the minimal operationally resolvable scale of $\gamma$
satisfies
\begin{equation}
\delta\gamma_{\min}
\;\gtrsim\;
\sqrt{\frac{4\Dstar\,\gamma}{T}} .
\label{eq:ou_bound}
\end{equation}
\end{proposition}

\paragraph{Assumptions.}
\begin{itemize}
\item[(A1)] Stationary correlations with finite correlation time $\gamma^{-1}$;
\item[(A2)] continuous monitoring without independent auxiliary probes or
externally injected resets;
\item[(A3)] local (quadratic) Fisher-information geometry.
\end{itemize}

\paragraph{Proof sketch.}
The distinguishability-rate envelope is set by the correlation time and bounds the
rate at which independent information can be accumulated.
Substituting the asymptotic form
$\FI_T(\gamma)\simeq T/(2\gamma)$ into the operational decision criterion yields
Eq.~\eqref{eq:ou_bound}.
\hfill$\square$

\paragraph{Interpretation.}
Proposition~\ref{prop:ou_bound} is epistemic.
It reflects correlation-limited information extraction through the specified
observation channel and does not imply any intrinsic discreteness, stochasticity,
or lower bound on $\gamma$ itself.

\subsection{Connection to the master inequality}

Equation~\eqref{eq:ou_bound} is a direct instantiation of the master inequality
(Proposition~\ref{prop:master_inequality}).
Here, the distinguishability-rate envelope is fixed by the correlation structure
of the process rather than by sampling frequency.
Increasing the sampling rate without modifying correlations does not increase
$\DistRate$.

\subsection{Falsifiability}

The rate-limited hypothesis for continuous monitoring would be falsified by an
experiment demonstrating Fisher information accumulation about $\gamma$ faster
than linear in $T$ without modifying the observation channel or introducing
additional independent degrees of freedom.
Such an observation would indicate either an unaccounted auxiliary channel or a
failure of the assumed stochastic description.

A direct diagnostic is to compare estimators built from
(i) marginal statistics alone and
(ii) correlation-aware statistics.
For stationary OU dynamics under assumptions (O1)--(O3), only correlation-aware
protocols can access $\gamma$.
Persistent marginal-only success would signal the presence of additional,
externally supplied information (e.g.\ calibrated knowledge of $D$ or extra
measured degrees of freedom).

\subsection{Relation to quantum monitoring}

Continuous classical monitoring provides a direct analogue of many quantum
measurement scenarios, including weak measurement and open-system dynamics.
In quantum systems, quantum Fisher information replaces $\FI_T$, and decoherence
induces noise-suppressed distinguishability rates.
These quantum instantiations are treated explicitly in
Section~\ref{sec:quantum_systems}.

% ============================================================
% End of sections/05_continuous_monitoring.tex
% ============================================================
