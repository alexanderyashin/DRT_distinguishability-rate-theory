% ============================================================
% sections/05_continuous_monitoring.tex  (FULL REPLACEMENT)
% ============================================================

\section{Continuous Monitoring}
\label{sec:continuous_monitoring}

Sections~\ref{sec:rate_limited_bounds} and
\ref{sec:classical_diffusion} addressed inference from discrete or effectively
\iid\ detection events. Many experimentally relevant scenarios, however, involve
\emph{continuous monitoring}, where information about a parameter is encoded
primarily in temporal correlations rather than in independent samples.
This section formalizes such intrinsically \noniid\ regimes within DRT.

\subsection{Correlated inference as a distinct regime}

In continuous monitoring, the observation channel produces a stochastic time
series $Y(t)$ whose samples are correlated. Consequently, Fisher information about
a parameter $\theta$ is generally \emph{not} determined by marginal distributions
alone. It resides in joint statistics and, in particular, in temporal correlation
functions.

Operationally, this implies that information accumulation is constrained by the
intrinsic correlation structure of the process. Treating correlated data streams
as \iid\ samples overestimates achievable resolution. DRT therefore treats
continuous monitoring as a distinct inference regime rather than as a limiting
case of \iid\ sampling.

\subsection{Ornstein--Uhlenbeck dynamics as a canonical model}

A minimal analytically tractable example is the stationary
Ornstein--Uhlenbeck (OU) process,
\begin{equation}
dX_t = -\gamma X_t\,dt + \sqrt{2D}\,dW_t ,
\end{equation}
where $\gamma>0$ is the relaxation rate and $D$ sets the noise strength.
We consider continuous observation of $X_t$ over a time window $T$, with the goal
of inferring $\gamma$.

\begin{lemma}[Marginal identifiability no-go for stationary OU]
\label{lem:ou_marginal_nogo}
Consider the stationary OU process with unknown diffusion strength $D$ treated as
a nuisance parameter. If inference is restricted to \emph{single-time} (marginal)
statistics of $X_t$ (i.e., ignoring temporal ordering and correlations), then
$\gamma$ is not identifiable: the marginal distribution determines only the ratio
$D/\gamma$ and cannot operationally fix the correlation time $\gamma^{-1}$.
Equivalently, all operationally usable information about the \emph{rate} $\gamma$
resides in temporal correlations.
\end{lemma}

\paragraph{Assumptions.}
\begin{itemize}
\item[(O1)] Stationary OU dynamics;
\item[(O2)] inference uses only the single-time marginal of $X_t$
(unordered samples; no temporal correlations);
\item[(O3)] $D$ is not assumed known a priori (nuisance parameter).
\end{itemize}

\paragraph{Proof sketch.}
In stationarity, $X_t$ is Gaussian with mean $0$ and variance $\mathrm{Var}(X_t)=D/\gamma$.
Therefore, the marginal distribution depends on parameters only through the
combination $D/\gamma$. Without additional information fixing $D$, the mapping
$(\gamma,D)\mapsto D/\gamma$ is many-to-one, so $\gamma$ cannot be identified from
single-time statistics alone. By contrast, the autocorrelation function
$\langle X_t X_{t+\tau}\rangle = (D/\gamma)\,e^{-\gamma \tau}$ contains a distinct
timescale $\gamma^{-1}$, making $\gamma$ accessible through temporal correlations.
\hfill$\square$

\paragraph{Interpretation.}
Lemma~\ref{lem:ou_marginal_nogo} is a precise no-go statement for this non-\iid\
regime: marginal statistics can fix a stationary \emph{scale} (here $D/\gamma$),
but not the \emph{rate} $\gamma$ that governs correlation decay. This is the
operational sense in which ``time is in correlations''.

\subsection{Correlation-limited information accumulation}

Because information resides in correlations, distinguishability accumulates only
over timescales exceeding the correlation time $\gamma^{-1}$. For continuous
monitoring of stationary OU dynamics, the accumulated Fisher information about
$\gamma$ scales asymptotically as
\begin{equation}
\FI_T(\gamma) \;\simeq\; \frac{T}{2\gamma} ,
\end{equation}
up to model-dependent prefactors fixed by the observation channel.

\begin{proposition}[OU correlation-limited bound]
\label{prop:ou_bound}
Under continuous monitoring of stationary OU dynamics satisfying
assumptions (O1)--(O3), the minimal operationally resolvable scale of $\gamma$
satisfies
\begin{equation}
\delta\gamma_{\min}
\;\gtrsim\;
\sqrt{\frac{4\Dstar\,\gamma}{T}} .
\label{eq:ou_bound}
\end{equation}
\end{proposition}

\paragraph{Assumptions.}
\begin{itemize}
\item[(A1)] Stationary OU correlations with finite correlation time $\gamma^{-1}$;
\item[(A2)] continuous monitoring without independent auxiliary probes;
\item[(A3)] local (quadratic) Fisher-information geometry.
\end{itemize}

\paragraph{Proof sketch.}
The Fisher information rate is set by the decay of correlations and is bounded by
the inverse correlation time. Substitution of the asymptotic form
$\FI_T(\gamma)\simeq T/(2\gamma)$ into the operational decision criterion
yields Eq.~\eqref{eq:ou_bound}.
\hfill$\square$

\paragraph{Interpretation.}
Proposition~\ref{prop:ou_bound} is epistemic. It reflects correlation-limited
information extraction through the specified observation channel and does not
imply any intrinsic lower bound on $\gamma$ itself.

\subsection{Connection to the master inequality}

Equation~\eqref{eq:ou_bound} is a direct instantiation of the master inequality
(Proposition~\ref{prop:master_inequality}). Here, the distinguishability-rate
envelope is fixed by the correlation structure of the process rather than by
sampling frequency. Increasing the sampling rate without altering correlations
does not increase $\DistRate$.

\subsection{Falsifiability}

The rate-limited hypothesis for continuous monitoring would be falsified by an
experiment demonstrating Fisher information accumulation about $\gamma$ faster
than linear in $T$ without modifying the observation channel or introducing
additional independent degrees of freedom. Such a result would indicate either
hidden information channels or a breakdown of the assumed stochastic dynamics.

A direct diagnostic is to compare estimators built from (i) marginal statistics
alone versus (ii) correlation-aware statistics. In the OU case, correlation-aware
protocols must dominate for estimating the rate $\gamma$ under assumptions
(O1)--(O3). Persistent marginal-only success at fixing $\gamma$ in that regime
would signal an unaccounted auxiliary channel (e.g.\ calibrated knowledge of $D$
or additional measured degrees of freedom).

\subsection{Relation to quantum monitoring}

Continuous classical monitoring provides the analogue of many quantum measurement
scenarios, including weak measurement and open-system dynamics. In quantum
systems, quantum Fisher information replaces $\FI_T$, and decoherence induces
noise-suppressed distinguishability rates. These quantum instantiations are
treated explicitly in Section~\ref{sec:quantum_systems}.

% ============================================================
% End of sections/05_continuous_monitoring.tex
% ============================================================
