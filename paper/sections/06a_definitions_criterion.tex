% ============================================================
% sections/06a_definitions_criterion.tex
% ============================================================

\section{Operational Definitions and Diagnostic Criterion}
\label{sec:definitions_criterion}

This subsection introduces the minimal operational definitions required to
separate inference-limited constraints from non-epistemic structural
constraints.
All statements in subsequent sections are explicitly scoped to the declared
class of observation channels and distinguishability geometry.

\subsection{Observation channels and distinguishability geometry}

\begin{definition}[Observation channel class]
A \emph{class of observation channels} $\mathcal{C}$ is a family of admissible
measurement or monitoring procedures sharing:
\begin{itemize}
\item[(C1)] a fixed event structure (classical or quantum),
\item[(C2)] a fixed parameterization $\theta \mapsto p(x|\theta)$ or
$\rho(\theta)$,
\item[(C3)] a well-defined local distinguishability geometry.
\end{itemize}
Throughout this work, $\mathcal{C}$ is assumed fixed unless stated otherwise.
\end{definition}

\begin{definition}[Local quadratic distinguishability]
An observation channel class $\mathcal{C}$ is said to admit a \emph{local
quadratic distinguishability geometry} if, for small parameter shifts,
the statistical distance admits the expansion
\begin{equation}
D(\theta,\theta+d\theta)^2 = I(\theta)\, d\theta^2 + o(d\theta^2),
\end{equation}
where $I(\theta)$ is the (classical or quantum) Fisher information.
\end{definition}

This assumption is standard in both classical and quantum estimation theory and
defines the scope of Fisher-reducible bounds.

\subsection{Fisher-reducible constraints}

\begin{definition}[Fisher-reducible constraint]
A resolution bound $\delta\theta_{\min}(I_T)$ is \emph{Fisher-reducible} within a
channel class $\mathcal{C}$ if it can be expressed solely as a function of the
accumulated Fisher information $I_T$ associated with $\mathcal{C}$.
\end{definition}

Such bounds arise entirely from local distinguishability and statistical
inference, independently of global structural features.

\subsection{Epistemic exhaustion}

\begin{definition}[Epistemic exhaustion]
A Fisher-reducible resolution bound is said to admit \emph{epistemic exhaustion}
within $\mathcal{C}$ if
\begin{equation}
\lim_{I_T \to \infty} \delta\theta_{\min}(I_T) = 0,
\end{equation}
while the observation channel class $\mathcal{C}$ and its event structure remain
fixed.
\end{definition}

Epistemic exhaustion corresponds to the idealized limit of unbounded resources
applied to the same observation model.

\begin{lemma}[Closure under epistemic exhaustion]
\label{lem:closure}
Within a fixed channel class $\mathcal{C}$, epistemic operations---including
increased statistics, reduced noise, or optimized estimators---can only relax
Fisher-reducible constraints and cannot modify the underlying event structure.
\end{lemma}

\paragraph{Remark.}
Lemma~\ref{lem:closure} explicitly excludes changes of channel class.
Statements beyond this scope are not addressed by DRT.

\subsection{Operational diagnostic criterion}

\begin{theorem}[Operational diagnostic criterion]
\label{thm:operational_criterion}
Within a fixed observation channel class $\mathcal{C}$:
\begin{itemize}
\item[(i)] if a constraint vanishes under epistemic exhaustion, it is epistemic;
\item[(ii)] if a constraint persists under epistemic exhaustion and cannot be
expressed as a Fisher-reducible bound, it reflects non-epistemic structure within
$\mathcal{C}$.
\end{itemize}
\end{theorem}

\paragraph{Scope.}
Theorem~\ref{thm:operational_criterion} is purely operational.
It does not assert global ontological claims and does not compare different
channel classes.

% ============================================================
% End of sections/06a_definitions_criterion.tex
% ============================================================
