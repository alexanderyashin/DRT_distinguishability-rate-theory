% ============================================================
% sections/09_conclusions.tex  (FULL REPLACEMENT)
% ============================================================

\section{Conclusions}
\label{sec:conclusions}

This work introduced \emph{Distinguishability Rate Theory} (DRT) as an operational
framework for analyzing limits of temporal, phase, and parameter resolution.
The central result is a reclassification: a broad and heterogeneous class of
reported bounds arises from finite rates of distinguishability accumulation
imposed by a given observation channel, rather than from microscopic dynamics or
ontological structure.

\subsection{Established results}

The analysis establishes the following results.

\paragraph{Inference as the operative constraint.}
Operational resolution is governed by the accumulation of Fisher or quantum
Fisher information under finite statistics, noise, correlations, and bandwidth.
Whenever distinguishability accumulates at a bounded rate, resolution limits
follow as an operational consequence of inference under the declared channel,
not as statements about the existence or structure of the parameter itself.

\paragraph{Rate-limited scaling constructions.}
Self-consistent rate arguments give rise to operational scaling
\emph{constructions}, including the $\Flux^{-1/3}$ scaling for photon-limited
diffusion and its generalization to anomalous transport.
These scalings depend exclusively on coarse, operational features of the
observation channel and do not constitute inference-supported bounds or
dynamical laws.

\paragraph{Non-\iid\ inference as a distinct regime.}
Continuous monitoring and correlated data streams constitute a qualitatively
distinct inference regime.
In such settings, information is encoded in temporal correlations, leading to
correlation-limited accumulation and linear-in-time asymptotics that cannot be
circumvented by increased sampling density alone.

\paragraph{Epistemic--ontological boundary.}
Quantum systems do not invalidate the rate-limited framework.
Resolution bounds reducible to local (quantum) Fisher information are epistemic
and relax under epistemic exhaustion.
Constraints that persist under epistemic exhaustion—such as those arising from
non-classical event structure—are identified as non-epistemic residues.
This criterion provides a necessary and sufficient stopping rule for
inference-based arguments.

\subsection{Conceptual contribution}

Beyond individual constructions and bounds, the primary contribution of DRT is
conceptual.
It supplies a precise diagnostic criterion for determining whether a claimed
limit reflects inference under finite resources or encodes irreducible
structural content of the theory.
By enforcing this distinction, DRT prevents the systematic misinterpretation of
operational limits as statements about the nature of time, phase, or physical
reality.

\subsection{Relation to existing approaches}

DRT does not replace estimation theory, quantum metrology, or foundational
analysis.
Instead, it reorganizes their results around a single operational primitive:
the rate of distinguishability accumulation.
This reorganization clarifies the assumptions and regimes under which familiar
bounds apply and explains their common operational origin across classical and
quantum domains.

\subsection{Explicit limitations}

The present work is intentionally restricted to local distinguishability
geometry and explicitly specified observation channels.
DRT does not address global state-space constraints, non-classical event
algebra, or foundational structures that lie outside inference theory.
Such features appear exclusively as non-epistemic residues and require
independent theoretical treatment.

\subsection{Outlook}

Future extensions may refine, but cannot overturn, the core classification
established here.
Multi-parameter inference, adaptive strategies, and strongly non-Markovian
environments modify distinguishability-rate envelopes and regime boundaries,
without altering the epistemic--ontological separation.

\subsection{Final statement}

Distinguishability Rate Theory reframes limits of resolution as statements about
information flow through observation channels rather than statements about the
structure of reality.
By identifying which constraints are inference-limited and which are genuinely
non-epistemic, it provides a unified operational language across classical and
quantum systems and delineates a precise boundary beyond which inference-based
reasoning cannot proceed.

% ============================================================
% End of sections/09_conclusions.tex
% ============================================================
