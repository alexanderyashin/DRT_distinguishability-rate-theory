% ============================================================
% sections/04_classical_diffusion.tex  (FULL REPLACEMENT)
% ============================================================

\section{Classical Diffusion}
\label{sec:classical_diffusion}

This section provides a concrete classical realization of the
rate-limited framework. We analyze inference of elapsed time from a
diffusive trajectory observed through a photon-limited channel.
No new stochastic physics is introduced. The purpose is to show how a
universal operational scaling \emph{construction} follows from
information-rate constraints and self-consistent closure alone.

\subsection{Model and observation channel}

We consider one-dimensional normal diffusion with diffusion coefficient
$D$, characterized by the mean-squared displacement
\begin{equation}
\langle x^2(t) \rangle = 2Dt .
\end{equation}
The system is observed via a Poissonian detection process with photon
flux $\Flux$. Each detection produces a noisy position estimate with
Gaussian point-spread uncertainty $\sigma_m$.

The observation channel is specified by:
\begin{itemize}
\item[(C1)] Poissonian (i.i.d.) photon counts with mean rate $\Flux$;
\item[(C2)] finite spatial resolution $\sigma_m$;
\item[(C3)] continuous-time latent diffusion dynamics.
\end{itemize}
The inferred parameter is the elapsed time $t$, accessed indirectly
through the statistics of measured positions.

\subsection{Information accumulation mechanism}

Each detection event contributes finite Fisher information about the
particle position and, indirectly, about elapsed time through the
diffusive growth of spatial uncertainty. Crucially, the Fisher
information gained per detection depends on the instantaneous
localization uncertainty, which itself evolves due to diffusion and
measurement backaction.

Operationally, the accumulated Fisher information about time can be
written as
\begin{equation}
\FI_T(t)
\;=\;
\int_0^T \DistRate(t')\,dt' ,
\end{equation}
where $\DistRate(t')$ is the instantaneous distinguishability rate.
This rate is bounded by the photon flux $\Flux$ and suppressed when the
diffusive spread $\sqrt{2Dt'}$ exceeds the measurement resolution
$\sigma_m$. The feedback between diffusion-induced uncertainty growth
and measurement-induced uncertainty reduction is the central mechanism
governing the resulting closure.

\subsection{Self-consistent localization and scaling}

The inference problem closes through a self-consistency condition.
Diffusion increases spatial uncertainty as $\sqrt{2Dt}$, while
measurements reduce uncertainty at a rate proportional to the photon
flux $\Flux$. The minimal resolvable time scale corresponds to a
fixed point at which information extraction balances uncertainty
generation.

\begin{theorem}[Self-consistent diffusion scaling]
\label{thm:phi_minus_one_third}
Consider normal diffusion observed through a Poisson-limited channel
with flux $\Flux$ and spatial resolution $\sigma_m$.
Under assumptions (C1)--(C3) and local distinguishability geometry,
the minimal operationally resolvable time scale satisfies
\begin{equation}
\delta t_{\min}
\;\sim\;
\left(
\frac{\sigma_m^2}{D^2\,\Flux}
\right)^{1/3}
\;\propto\;
\Flux^{-1/3}.
\label{eq:phi_minus_one_third}
\end{equation}
\end{theorem}

\paragraph{Assumptions.}
\begin{itemize}
\item[(A1)] Poissonian detection with finite flux $\Flux$;
\item[(A2)] Gaussian measurement noise of fixed width $\sigma_m$;
\item[(A3)] validity of local (quadratic) Fisher-information geometry;
\item[(A4)] existence of a self-consistent fixed point where diffusion
and information gain balance.
\end{itemize}

\paragraph{Proof sketch.}
Diffusion increases localization uncertainty as $\sqrt{2Dt}$, reducing
the Fisher information per detection, while measurements decrease
uncertainty at a rate proportional to $\Flux$.
Equating the diffusion-induced growth of uncertainty with the
measurement-induced reduction yields a fixed-point condition.
Solving this condition produces the cubic-root scaling.
A detailed derivation is given in Appendix~A.
\hfill$\square$

\paragraph{Interpretation and epistemic status.}
Theorem~\ref{thm:phi_minus_one_third} is a
\emph{self-consistent fixed-point construction} (Class~0A; non-inference).
It constrains achievable time resolution under the specified
observation channel and does not represent a scaling bound established
by decision-based inference.
No intrinsic granularity or discreteness of time is implied.

\subsection{Numerical consistency checks}

The $\Flux^{-1/3}$ scaling constructed above is accompanied by frozen
Monte Carlo simulations combining explicit diffusive trajectories with
Poisson detection and Gaussian measurement noise.
These simulations implement the specified observation model and serve
as \emph{numerical consistency checks} of the analytic construction and
the slope-fitting pipeline.
They do not constitute independent inference evidence for the scaling
exponent.

The frozen Evidence Pack yields fitted slopes consistent with the
construction exponent $-1/3$ across multiple decades in $\Flux$
(Figure~\ref{fig:phi_scaling}).
Robustness against statistical fluctuations and finite-$\Flux$ effects
is illustrated by multi-seed slope statistics
(Figure~\ref{fig:phi_hist}).
All numerical protocols and random seeds are fixed and documented
(Appendices~D and~E).

\subsection{Regime of validity}

The construction applies in the inference-limited regime where:
\begin{itemize}
\item[(R1)] photon statistics are Poissonian;
\item[(R2)] observation times are sufficient to reach the
self-consistent fixed point;
\item[(R3)] systematic effects (drift, miscalibration) are negligible.
\end{itemize}
Deviations from the predicted envelope diagnose finite-resource
(pre-asymptotic) effects or violations of the assumed observation
channel, not a failure of diffusive dynamics.

\paragraph{Falsifiability.}
Observation of a scaling steeper than $\Flux^{-1/3}$ under assumptions
(C1)--(C3) would falsify the applicability of the
self-consistent fixed-point construction in this setting and, with it,
the rate-limited hypothesis under the stated channel assumptions.

\subsection{Extension to anomalous diffusion}

The same self-consistent closure logic applies to anomalous transport
with mean-squared displacement $\mathrm{MSD}(t)\sim t^\alpha$.

\begin{theorem}[Anomalous diffusion / CTRW generalization]
\label{thm:ctrw_generalization}
Assume anomalous transport with $\mathrm{MSD}(t)\sim t^\alpha$ for
$0<\alpha<2$ under the same Poisson-limited observation channel.
Then the minimal operationally resolvable time scale obeys
\begin{equation}
\delta t_{\min}\;\propto\;\Flux^{-1/(2+\alpha)}.
\end{equation}
\end{theorem}

\paragraph{Status.}
Theorem~\ref{thm:ctrw_generalization} is epistemic and should be
interpreted as an analytic generalization of the fixed-point
construction.
It reduces to Theorem~\ref{thm:phi_minus_one_third} for $\alpha=1$ and is
derived formally in Appendix~B.
The accompanying CTRW $\alpha$-sweep
(Figure~\ref{fig:ctrw_alpha}) is generated by an exponent-imposed CTRW
simulator (Class~0B; non-inference) and serves as a numerical
consistency check of the construction and analysis pipeline rather than
as inference validation.

% ============================================================
% End of sections/04_classical_diffusion.tex
% ============================================================
