% ============================================================
% sections/04_classical_diffusion.tex
% ============================================================

\section{Classical Diffusion}
\label{sec:classical_diffusion}

This section provides a concrete classical realization of the rate-limited
framework by analyzing inference of elapsed time from a diffusive trajectory
under photon-limited observation. No new stochastic physics is introduced.
Rather, the goal is to demonstrate how a universal operational scaling emerges
from information-rate constraints and self-consistent inference alone.

\subsection{Model and observation channel}

We consider a particle undergoing one-dimensional normal diffusion with
diffusion coefficient $D$, such that the mean-square displacement satisfies
\begin{equation}
\langle x^2(t) \rangle = 2Dt .
\end{equation}
The particle is observed through a Poissonian detection process with photon
flux $\Flux$. Each detection yields a noisy position estimate with Gaussian
point-spread uncertainty $\sigma_m$.

The observation channel is therefore characterized by:
(i) i.i.d.\ Poisson count statistics,
(ii) finite spatial resolution $\sigma_m$, and
(iii) continuous temporal evolution of an unobserved latent trajectory.
The parameter of interest is the elapsed time $t$, inferred indirectly through
the observed spatial statistics.

\subsection{Information accumulation under diffusion}

Each detection event contributes a finite amount of Fisher information about
the particle’s position, and indirectly about elapsed time through the
diffusive growth of spatial uncertainty. Crucially, the effective information
gain per detection depends on the current localization uncertainty, which
itself evolves as information accumulates.

Operationally, the accumulated Fisher information about time can be written as
\begin{equation}
\FI_T(t)
\;=\;
\int_0^T \DistRate(t')\,dt' ,
\end{equation}
where the instantaneous distinguishability rate $\DistRate$ is bounded by the
photon flux $\Flux$ and suppressed when the diffusive spread
$\sqrt{2Dt'}$ becomes large compared to the measurement resolution
$\sigma_m$. This feedback between diffusion-driven uncertainty growth and
measurement-driven uncertainty reduction is the essential mechanism.

\subsection{Self-consistent localization and scaling law}

The inference problem closes through a self-consistency condition. At short
times, diffusion broadens the position distribution as $\sqrt{2Dt}$, while
measurements reduce uncertainty at a rate proportional to the photon flux
$\Flux$. The minimal resolvable time scale is determined by the point at which
the rate of information extraction balances the rate at which new uncertainty
is injected by diffusion.

\begin{theorem}[Self-consistent diffusion scaling]
\label{thm:phi_minus_one_third}
Consider normal diffusion observed through a Poissonian detection channel with
flux $\Flux$ and finite spatial resolution $\sigma_m$.
Under the assumptions of local distinguishability geometry and finite-rate
information accumulation, the minimal operationally resolvable time scale obeys
\begin{equation}
\delta t_{\min}
\;\sim\;
\left(
\frac{\sigma_m^2}{D^2\,\Flux}
\right)^{1/3}
\;\propto\;
\Flux^{-1/3}.
\label{eq:phi_minus_one_third}
\end{equation}
\end{theorem}

\paragraph{Assumptions.}
(i) Poissonian (i.i.d.) detection statistics with finite flux $\Flux$;  
(ii) Gaussian measurement noise with fixed width $\sigma_m$;  
(iii) validity of local (quadratic) Fisher-information geometry;  
(iv) existence of a self-consistent fixed point where information gain and
diffusion-induced uncertainty balance.

\paragraph{Proof sketch.}
The proof follows a fixed-point argument. Diffusion increases spatial
uncertainty as $\sqrt{2Dt}$, reducing the Fisher information extractable per
detection, while measurements reduce uncertainty at a rate proportional to
$\Flux$. Equating these competing rates yields the cubic-root scaling.
A detailed derivation is provided in Appendix~A.
\hfill$\square$

\paragraph{Status.}
Theorem~\ref{thm:phi_minus_one_third} is epistemic: it constrains achievable time
resolution under the specified observation channel. It does not assert any
fundamental discreteness or granularity of time itself.

\subsection{Numerical evidence}

The $\Flux^{-1/3}$ scaling is confirmed by Monte Carlo simulations combining
explicit diffusive trajectories with Poisson detection and Gaussian measurement
noise. The frozen Evidence Pack yields a fitted slope consistent with $-1/3$
over multiple decades in $\Flux$ (Figure~\ref{fig:phi_scaling}).

Robustness against statistical fluctuations and finite-$\Flux$ effects is
demonstrated by multi-seed slope statistics
(Figure~\ref{fig:phi_hist}). All numerical protocols and random seeds are fixed
and documented (Appendices~D and~E).

\subsection{Regime of validity and finite-$\Flux$ systematics}

The $\Flux^{-1/3}$ law holds in the inference-limited regime where:
(i) photon statistics are Poissonian,
(ii) the observation time is sufficient to reach the self-consistent fixed
point, and
(iii) systematic effects such as drift or miscalibration are negligible.

Deviations from the predicted scaling diagnose either finite-resource
(pre-asymptotic) effects or violations of the assumed observation channel, rather
than a breakdown of diffusion dynamics. Such deviations therefore act as
diagnostic signals rather than evidence for ontological limits.

\paragraph{Falsifiability.}
Persistent experimental observation of a scaling steeper than $\Flux^{-1/3}$
under the stated channel assumptions would falsify
Theorem~\ref{thm:phi_minus_one_third} and, by extension, the rate-limited
hypothesis for this setting.

\subsection{Connection to anomalous diffusion}

The same self-consistency logic extends to anomalous diffusion with
$\mathrm{MSD}\sim t^\alpha$, yielding the generalized scaling
$\delta t_{\min}\propto\Flux^{-1/(2+\alpha)}$.
This generalization is derived in Appendix~B and supported numerically by the
CTRW $\alpha$-sweep
(Figure~\ref{fig:ctrw_alpha}). The normal diffusion case analyzed here
corresponds to $\alpha=1$ and serves as the baseline against which anomalous
transport regimes are compared.

\begin{theorem}[Anomalous diffusion / CTRW generalization]
\label{thm:ctrw_generalization}
Assume anomalous transport with mean-squared displacement $\mathrm{MSD}(t)\sim t^\alpha$
for $0<\alpha<2$ under the same Poisson-limited observation channel as above.
Then the self-consistent inference logic yields the generalized scaling law
\begin{equation}
\delta t_{\min}\;\propto\;\Flux^{-1/(2+\alpha)}.
\end{equation}
\end{theorem}


% ============================================================
% End of sections/04_classical_diffusion.tex
% ============================================================
