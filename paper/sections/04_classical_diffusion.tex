% ============================================================
% sections/04_classical_diffusion.tex  (FULL REPLACEMENT)
% ============================================================

\section{Classical Diffusion}
\label{sec:classical_diffusion}

This section provides a concrete classical realization of the
rate-limited framework. We analyze inference of elapsed time from a
diffusive trajectory observed through a photon-limited localization
channel. No new stochastic physics is introduced. The purpose is to show
how an operational scaling \emph{construction} follows from
information-rate constraints together with a self-consistent closure
(Class~0A), and how this differs from inference-only optimum-time
regimes.

\subsection{Model and observation channel}

We consider one-dimensional normal diffusion with diffusion coefficient
$D$, characterized by the mean-squared displacement
\begin{equation}
\langle x^2(t) \rangle = 2Dt .
\end{equation}
The system is observed via a Poissonian detection process with photon
flux $\Flux$. Each detection produces a noisy position sample with
Gaussian point-spread uncertainty $\sigma_m$.

The observation channel is specified by:
\begin{itemize}
\item[(C1)] Poissonian photon counts with mean rate $\Flux$;
\item[(C2)] finite spatial resolution $\sigma_m$;
\item[(C3)] continuous-time latent diffusion dynamics.
\end{itemize}
The inferred parameter is the elapsed time $t$, accessed indirectly
through the time-dependence of the observed variance.

\subsection{Information accumulation mechanism}

Each detection event contributes finite Fisher information about elapsed
time through the diffusive growth of the observable variance. In the
Gaussian-local regime used in this paper, the Fisher information about
time under the Poisson-limited channel can be written in rate form,
\begin{equation}
\FI_T(t)
\;=\;
\int_0^{t} \DistRate(t')\,dt' ,
\end{equation}
where $\DistRate(t')$ is the instantaneous distinguishability (FI)
accumulation rate, bounded by the photon flux $\Flux$ and suppressed when
the observable variance is large.

\paragraph{Guard against a common surrogate.}
We explicitly \emph{do not} model ``effective estimator variance'' as
$2Dt+\sigma_m^2/(\Flux t)$ and then impose a fixed-point relation
$\delta t \sim \sigma_{\mathrm{est}}^2(\delta t)/(2D)$.
That chain algebraically leads to $\delta t \propto \Flux^{-1/2}$ and
corresponds to an inference-only optimum-time regime, not to the Class~0A
closure used here. Our $\Flux^{-1/3}$ scaling is derived solely from the
estimator-variance closure in Appendix~\ref{app:phi_minus_one_third}.

\subsection{Self-consistent closure and scaling (Class 0A)}

The inference problem closes through a self-consistency condition:
finite photon flux limits how quickly time-information can be harvested,
and the resulting estimator variance feeds back into the effective
observation window. This produces a closure (Class~0A) rather than a
CRLB-style extremization.

\begin{theorem}[Self-consistent diffusion scaling (Class 0A)]
\label{thm:phi_minus_one_third}
Consider normal diffusion observed through a Poisson-limited localization
channel with flux $\Flux$ and spatial resolution $\sigma_m$.
Under assumptions (C1)--(C3) and local distinguishability geometry, the
minimal operationally resolvable time scale satisfies
\begin{equation}
\delta t_{\min}
\;\sim\;
\left(
\frac{\sigma_m^4}{D^2\,\Flux}
\right)^{1/3}
\;\propto\;
\Flux^{-1/3}.
\label{eq:phi_minus_one_third}
\end{equation}
\end{theorem}

\paragraph{Assumptions.}
\begin{itemize}
\item[(A1)] Poissonian detection with finite flux $\Flux$;
\item[(A2)] Gaussian measurement noise of fixed width $\sigma_m$;
\item[(A3)] validity of local (quadratic) Fisher-information geometry;
\item[(A4)] existence of a self-consistent closure identifying the
effective observation window with the resolution scale (Class~0A).
\end{itemize}

\paragraph{Proof sketch (pointer only).}
The proof is a direct unpacking of the estimator-variance closure:
(i) define an explicit estimator of elapsed time via the observed
variance, (ii) compute its finite-sample variance under $N\sim
\mathrm{Poisson}(\Flux t)$ detections, and (iii) impose the closure
$\delta t^2 \sim \mathrm{Var}[\widehat{t}]$ at $t=\delta t$.
This yields $\delta t^3 \propto \sigma_m^4/(D^2\Flux)$ and hence
$\delta t\propto \Flux^{-1/3}$.
The full derivation (including the FI form consistent with the closure)
is given in Appendix~\ref{app:phi_minus_one_third}.
\hfill$\square$

\paragraph{Interpretation and epistemic status.}
Theorem~\ref{thm:phi_minus_one_third} is a
\emph{self-consistent closure construction} (Class~0A; non-inference).
It constrains achievable time resolution under the specified channel and
does not represent a scaling law established by inference optimality.
No intrinsic discreteness of time is implied.

\subsection{Numerical consistency checks}

The $\Flux^{-1/3}$ scaling constructed above is accompanied by frozen
Monte Carlo simulations combining explicit diffusive trajectories with
Poisson detection and Gaussian measurement noise. These simulations
implement the specified observation model and serve as \emph{numerical
consistency checks} of the analytic construction and the slope-fitting
pipeline. They do not constitute independent inference evidence for the
exponent.

The frozen Evidence Pack yields fitted slopes consistent with the
construction exponent $-1/3$ across multiple decades in $\Flux$
(Figure~\ref{fig:phi_scaling}). Robustness against statistical
fluctuations and finite-$\Flux$ effects is illustrated by multi-seed
slope statistics (Figure~\ref{fig:phi_hist}). All numerical protocols and
random seeds are fixed and documented (Appendices~D and~E).

\subsection{Regime of validity}

The construction applies in the closure-limited regime where:
\begin{itemize}
\item[(R1)] photon statistics are Poissonian;
\item[(R2)] observation windows are sufficient for the closure to be
well-posed (Appendix~\ref{app:phi_minus_one_third});
\item[(R3)] systematic effects (drift, miscalibration) are negligible;
\item[(R4)] the closure occurs while $2D\,\delta t \ll \sigma_m^2$ so that
the observable variance is measurement-dominated at the fixed point.
\end{itemize}
Deviations from the predicted envelope diagnose pre-asymptotic effects
or violations of the assumed observation channel, not a failure of
diffusive dynamics.

\paragraph{Falsifiability.}
Observation of a scaling steeper than $\Flux^{-1/3}$ under assumptions
(C1)--(C3) would falsify the applicability of the Class~0A closure in
this setting and, with it, the stated rate-limited hypothesis under the
given channel assumptions.

\subsection{Extension to anomalous diffusion}

The same closure logic can be written for anomalous transport with
$\mathrm{MSD}(t)\sim t^\alpha$.

\begin{theorem}[Anomalous diffusion / CTRW generalization (Class 0B status)]
\label{thm:ctrw_generalization}
Assume anomalous transport with $\mathrm{MSD}(t)\sim t^\alpha$ for
$0<\alpha<2$ under the same Poisson-limited observation channel.
Then the closure construction yields
\begin{equation}
\delta t_{\min}\;\propto\;\Flux^{-1/(2+\alpha)}.
\end{equation}
\end{theorem}

\paragraph{Status.}
Theorem~\ref{thm:ctrw_generalization} is to be interpreted as an analytic
extension of the fixed-point/closure logic.
It reduces to Theorem~\ref{thm:phi_minus_one_third} for $\alpha=1$ and is
derived formally in Appendix~\ref{app:ctrw_derivations}.
The accompanying CTRW $\alpha$-sweep (Figure~\ref{fig:ctrw_alpha}) is
generated by an exponent-imposed CTRW simulator (Class~0B; non-inference)
and serves as a numerical consistency check of the construction and
analysis pipeline rather than as inference validation.

% ============================================================
% End of sections/04_classical_diffusion.tex
% ============================================================
