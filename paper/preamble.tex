% ============================================================
% preamble.tex
% ============================================================
% LaTeX preamble for the DRT paper.
%
% IMPORTANT:
%  - This file MUST NOT contain \documentclass or \begin{document}.
%  - main.tex is the only compilation entry point.
%
% Notes:
%  - The package physics.sty is not available in the current TeXLive
%    environment. We use a minimal, portable substitute instead.
% ============================================================

% ------------------------------------------------------------
% Encoding and fonts
% ------------------------------------------------------------
\usepackage[T1]{fontenc}
\usepackage[utf8]{inputenc}
\usepackage{lmodern}

% ------------------------------------------------------------
% Page layout
% ------------------------------------------------------------
\usepackage{geometry}
\geometry{
  left=2.5cm,
  right=2.5cm,
  top=2.8cm,
  bottom=3.0cm
}

% ------------------------------------------------------------
% Typography and spacing
% ------------------------------------------------------------
\usepackage{microtype}
\usepackage{setspace}
\onehalfspacing

\setlength{\parindent}{0pt}
\setlength{\parskip}{0.6em}

% Overfull line mitigation (global, minimal)
\setlength{\emergencystretch}{2em}
\hfuzz=0.5pt

% ------------------------------------------------------------
% Mathematics
% ------------------------------------------------------------
\usepackage{amsmath}
\usepackage{amssymb}
\usepackage{amsthm}
\usepackage{bm}
\usepackage{mathtools}

% Theorem environments (used sparingly, operational focus)
\newtheorem{theorem}{Theorem}
\newtheorem{lemma}{Lemma}
\newtheorem{proposition}{Proposition}
\newtheorem{corollary}{Corollary}

% ------------------------------------------------------------
% Minimal "physics" substitutes (portable)
% ------------------------------------------------------------
\usepackage{braket}

% Derivatives
\newcommand{\dv}[2]{\frac{d #1}{d #2}}
\newcommand{\ddv}[2]{\frac{d^2 #1}{d #2^2}}
\newcommand{\pdv}[2]{\frac{\partial #1}{\partial #2}}
\newcommand{\pddv}[2]{\frac{\partial^2 #1}{\partial #2^2}}

% Absolute value and norm
\newcommand{\abs}[1]{\left\lvert #1 \right\rvert}
\newcommand{\norm}[1]{\left\lVert #1 \right\rVert}

% Expectation value (classical)
\newcommand{\expval}[1]{\left\langle #1 \right\rangle}

% ------------------------------------------------------------
% Figures and tables
% ------------------------------------------------------------
\usepackage{graphicx}
\usepackage{subcaption}
\usepackage{booktabs}
\usepackage{longtable}

% External PDF inclusion (figures generated from Evidence Pack)
\usepackage{pdfpages}

% ------------------------------------------------------------
% References and hyperlinks
% ------------------------------------------------------------
\usepackage{hyperref}
\hypersetup{
  colorlinks=true,
  linkcolor=blue,
  citecolor=blue,
  urlcolor=blue,
  pdfauthor={Alexander Yashin},
  pdftitle={Distinguishability as a Physical Primitive}
}

% ------------------------------------------------------------
% Lists
% ------------------------------------------------------------
\usepackage{enumitem}
\setlist{noitemsep, topsep=0.3em}

% ------------------------------------------------------------
% Utilities
% ------------------------------------------------------------
\usepackage{xcolor}

% ------------------------------------------------------------
% Custom commands (minimal, semantic only)
% ------------------------------------------------------------

% Distinguishability rate
\newcommand{\DistRate}{v_{\mathrm{D}}}

% Fisher Information
\newcommand{\FI}{\mathcal{I}}

% Quantum Fisher Information
\newcommand{\QFI}{\mathcal{F}_{Q}}

% Observation flux (photon / event rate)
\newcommand{\Flux}{\Phi}

% Minimal resolvable time
\newcommand{\deltat}{\delta t_{\min}}

% ------------------------------------------------------------
% Document hygiene
% ------------------------------------------------------------
\sloppy
\raggedbottom

% ============================================================
% End of preamble.tex
% ============================================================
