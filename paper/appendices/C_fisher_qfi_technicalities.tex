% ============================================================
% appendices/C_fisher_qfi_technicalities.tex  (FULL REPLACEMENT)
% ============================================================

\section{Fisher and Quantum Fisher Information: Technical Background}
\label{app:fisher_qfi}

This appendix collects the technical foundations underlying the
distinguishability-rate formalism used throughout the paper.
Its role is purely formal: to render all claims involving Fisher
information (FI), quantum Fisher information (QFI), and their operational
interpretation explicit, precise, and reviewer-verifiable, without
interrupting the main text.

\subsection{Classical Fisher information}

Let $Y$ denote an observation drawn from a probability density
$p(y|\theta)$ depending smoothly on a parameter $\theta$.
The (classical) Fisher information is defined as
\begin{equation}
\FI(\theta)
\;=\;
\mathbb{E}_\theta\!\left[
\left(\partial_\theta \ln p(Y|\theta)\right)^2
\right].
\end{equation}
Operationally, $\FI(\theta)$ defines the local quadratic form governing
the statistical distinguishability of nearby parameter values.

For $N$ independent and identically distributed observations,
Fisher information is additive,
\begin{equation}
\FI_N(\theta) = N\,\FI_1(\theta).
\end{equation}
This additivity holds only in the \iid\ regime and is not assumed
elsewhere in the paper.

\subsection{Decision threshold and operational resolution}

Throughout, operational resolution is defined relative to a fixed,
dimensionless distinguishability threshold $\Dstar$.
A parameter offset $\delta\theta$ is said to be operationally resolvable
when
\begin{equation}
\delta\theta^2\, \FI_T(\theta) \;\ge\; 2\Dstar ,
\label{eq:decision_threshold_app}
\end{equation}
where $\FI_T(\theta)$ denotes the Fisher information accumulated over an
observation window of duration $T$.

The numerical value of $\Dstar$ encodes the chosen decision rule
(e.g.\ fixed error probability or likelihood-ratio threshold).
Changing $\Dstar$ rescales $\delta\theta_{\min}$ by a constant factor
only and does not affect scaling exponents, regime classification, or
falsifiability.

\paragraph{Class distinction (non-negotiable).}
Relations of the form
$\delta\theta_{\min}\propto \FI_T^{-1/2}$
characterize \emph{inference-only} resolution limits (Class~I), where the
observation window $T$ is an independently selectable control parameter.
Such relations do \emph{not} determine, predict, or constrain the
scaling exponents arising from self-consistent closures (Class~0A), in
which the observation window is identified with the resolution scale
itself.
No exponent appearing in a Class~0A construction is obtained by
extremizing Fisher information or by differentiating any surrogate
variance with respect to $T$.

\subsection{Information accumulation rate}

In non-\iid\ or continuous-monitoring settings, Fisher information may
accumulate continuously in time. When applicable, we write
\begin{equation}
\FI_T(\theta)
\;=\;
\int_0^T \dot{\FI}(t;\theta)\,dt ,
\end{equation}
where $\dot{\FI}(t;\theta)$ denotes an \emph{information accumulation
rate envelope}.

Crucially, $\dot{\FI}$ is not assumed to be a fundamental dynamical
quantity. It represents an upper bound on the rate at which
distinguishability can be extracted from the observation channel.
Noise, finite bandwidth, correlations, or decoherence suppress
$\dot{\FI}$ and thereby impose operational limits even for arbitrarily
long observation times.

\subsection{Quantum Fisher information}

For a smooth family of quantum states $\rho_\theta$, the quantum Fisher
information $\QFI(\theta)$ is defined by
\begin{equation}
\QFI(\theta)
\;=\;
\mathrm{Tr}\!\left[\rho_\theta L_\theta^2\right],
\end{equation}
where the symmetric logarithmic derivative $L_\theta$ satisfies
\begin{equation}
\partial_\theta \rho_\theta
\;=\;
\frac{1}{2}\left(L_\theta \rho_\theta + \rho_\theta L_\theta\right).
\end{equation}
Equivalently, $\QFI(\theta)$ is the supremum of classical Fisher
information over all admissible POVMs.

\paragraph{Accumulated QFI.}
To avoid ambiguity between local and time-integrated quantities, we
denote by $\mathcal{I}^{(Q)}_T(\theta)$ the \emph{accumulated} quantum
Fisher information available after an observation protocol of duration
$T$.
The operational decision bound becomes
\begin{equation}
\delta\theta_{\min}
\;\ge\;
\sqrt{\frac{2\Dstar}{\mathcal{I}^{(Q)}_T(\theta)}} .
\label{eq:qcrb_app}
\end{equation}
Within DRT, $\mathcal{I}^{(Q)}_T$ constrains inference only and carries no
ontological implication for $\theta$.

\subsection{Multiparameter geometry}

For a vector parameter $\boldsymbol{\theta}$, Fisher information
generalizes to a positive semidefinite matrix $\FI_{ij}$.
For a direction $u$ in parameter space, the inference-limited resolution
scale is
\begin{equation}
\delta_{\mathrm{inf}}(u;T)
\;=\;
\sqrt{\frac{2\Dstar}{u^{\mathsf T}\FI_T u}} .
\end{equation}
This defines the local Fisher ellipsoid.
Meeting-point behavior arises when this inference-limited scale matches
a dynamical or geometric scale intrinsic to the system.

\subsection{QFI under noise and decoherence}

In open quantum systems, accumulated QFI is generically suppressed.
For Markovian dephasing with rate $\Gamma$, an envelope of the form
\begin{equation}
\mathcal{I}^{(Q)}_T(\theta)
\;\le\;
\int_0^T
\dot{\QFI}_{\mathrm{ideal}}(t)\,e^{-2\Gamma t}\,dt
\end{equation}
holds. This suppression leads to saturation of accumulated
distinguishability and underlies the universal meeting-point behavior
observed in Ramsey and Mach--Zehnder interferometry.

\subsection{Fisher-reducible versus non-Fisher-reducible constraints}

The central technical distinction of the paper is the following:
\begin{itemize}
\item \textbf{Fisher-reducible constraints} are expressible solely as
bounds on $\FI_T$ or $\mathcal{I}^{(Q)}_T$ and vanish under epistemic
exhaustion ($\FI_T\!\to\!\infty$).
\item \textbf{Non-Fisher-reducible constraints} depend on global
algebraic, compatibility, or event-structure properties and persist even
as $\FI_T\!\to\!\infty$.
\end{itemize}
Only constraints of the second type qualify as ontological residues in
the sense of Section~\ref{sec:ontological_residues}.

\subsection{Scope and limitations}

All FI- and QFI-based results are local in parameter space and asymptotic
in nature. Global estimation problems, non-smooth parameterizations, and
constraints unrelated to distinguishability geometry lie outside the
scope of Distinguishability-Rate Theory and are not addressed here.

% ============================================================
% End of appendices/C_fisher_qfi_technicalities.tex
% ============================================================
