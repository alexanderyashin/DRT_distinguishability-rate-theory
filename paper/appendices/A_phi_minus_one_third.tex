% ============================================================
% appendices/A_phi_minus_one_third.tex  (FULL REPLACEMENT)
% ============================================================

\section{Self-Consistent Derivation of the
\texorpdfstring{$\boldsymbol{\Phi^{-1/3}}$}{Phi^{-1/3} Scaling}}
\label{app:phi_minus_one_third}

This appendix provides an explicit self-consistent derivation of the
cubic-root scaling
$\delta t_{\min}\propto \Phi^{-1/3}$ for photon-limited temporal localization of a
diffusive particle.
The derivation is purely operational and inference-based.
No microscopic assumptions are invoked beyond normal diffusion, Poisson
detection statistics, and finite measurement noise.

\subsection{Setup and assumptions}

We consider one-dimensional Brownian motion with diffusion coefficient $D$,
\begin{equation}
\langle x^2(t)\rangle = 2Dt,
\end{equation}
observed through a Poisson point process of detection events with constant photon
flux $\Phi$.
Each detection yields an independent position estimate with Gaussian
point-spread uncertainty $\sigma_m$.

The assumptions are:
\begin{enumerate}
\item Normal diffusion with finite $D$ and no drift.
\item Poissonian detection statistics with rate $\Phi$.
\item Independent measurement noise with variance $\sigma_m^2$ per detection.
\item Local inference governed by a fixed distinguishability threshold $D^\ast$
(Appendix~\ref{app:fisher_qfi}).
\end{enumerate}

\subsection{Operational localization variance}

Let $\sigma^2(t)$ denote the effective localization variance conditioned on the
measurement record accumulated up to time $t$.
Two competing mechanisms determine $\sigma^2(t)$:
\begin{itemize}
\item \textbf{Diffusive spreading:} uncertainty increases as $2Dt$.
\item \textbf{Measurement-induced reduction:} information accumulates at rate
$\Phi$, reducing variance as $\sigma_m^2/(\Phi t)$.
\end{itemize}

To leading operational order, the localization variance is therefore
\begin{equation}
\sigma^2(t)\;\sim\; 2Dt \;+\; \frac{\sigma_m^2}{\Phi t}.
\label{eq:sigma_balance}
\end{equation}
This expression captures the feedback between stochastic spreading and finite-rate
information acquisition.

\subsection{Self-consistent fixed point}

The characteristic inference time scale is obtained by minimizing
$\sigma^2(t)$ with respect to $t$.
Differentiating Eq.~\eqref{eq:sigma_balance} gives
\begin{equation}
\frac{d\sigma^2}{dt}
= 2D - \frac{\sigma_m^2}{\Phi t^2}.
\end{equation}
Setting the derivative to zero yields the fixed point
\begin{equation}
t_\ast
\;\sim\;
\left(\frac{\sigma_m^2}{D\,\Phi}\right)^{1/3}.
\label{eq:t_star}
\end{equation}
This time scale marks the balance between diffusion-induced uncertainty growth and
measurement-induced uncertainty reduction.

\subsection{Temporal distinguishability}

Temporal distinguishability arises because the diffusive state depends on time.
For small temporal offsets $\delta t$, the Fisher information about time scales
inversely with the localization variance,
\begin{equation}
I_T(t) \;\sim\; \frac{t}{\sigma^2(t)},
\label{eq:fi_scaling}
\end{equation}
up to dimensionless constants that do not affect scaling.
This reflects the fact that resolving time requires resolving changes in the
diffusive width over the observation interval.

Evaluating Eq.~\eqref{eq:fi_scaling} at the self-consistent scale $t_\ast$ yields
\begin{equation}
I_T(t_\ast)
\;\sim\;
\frac{t_\ast}{\sigma^2(t_\ast)}
\;\sim\;
\frac{\Phi^{2/3}}{D^{1/3}\sigma_m^{4/3}},
\end{equation}
where numerical prefactors have been suppressed.

\subsection{Operational time resolution}

The minimal resolvable temporal offset $\delta t_{\min}$ follows from the
decision criterion
\begin{equation}
\delta t^2\, I_T \;\gtrsim\; 2D^\ast.
\label{eq:decision_again}
\end{equation}
Substituting the scaling of $I_T(t_\ast)$ gives
\begin{equation}
\delta t_{\min}
\;\sim\;
\left(\frac{\sigma_m^2}{D^2\,\Phi}\right)^{1/3},
\end{equation}
or equivalently,
\begin{equation}
\delta t_{\min}\propto \Phi^{-1/3}.
\end{equation}

\subsection{Interpretation}

The cubic-root scaling is not a consequence of dimensional analysis alone.
It emerges from a self-consistent balance between:
\begin{enumerate}
\item stochastic diffusive spreading,
\item Poisson-limited information acquisition, and
\item a fixed operational distinguishability threshold.
\end{enumerate}
Changing any of these ingredients modifies the scaling.

\subsection{Epistemic status and validity}

The $\Phi^{-1/3}$ law is epistemic.
It vanishes in the limits $\Phi\to\infty$ or $\sigma_m\to 0$, reflecting improved
inference rather than any intrinsic discreteness of time.
Its validity is restricted to the inference-limited regime; pre-asymptotic
effects or alternative observation channels modify the effective behavior.

\subsection{Connection to generalizations}

For anomalous diffusion with $\mathrm{MSD}\sim t^\alpha$, the same
self-consistency logic yields
$\delta t_{\min}\propto\Phi^{-1/(2+\alpha)}$, as derived in
Appendix~B.
The normal diffusion case corresponds to $\alpha=1$ and serves as the canonical
baseline.

% ============================================================
% End of appendices/A_phi_minus_one_third.tex
% ============================================================
