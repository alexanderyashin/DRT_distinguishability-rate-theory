% ============================================================
% appendices/A_phi_minus_one_third.tex  (FULL REPLACEMENT)
% ============================================================

\section{Self-Consistent Fixed-Point Construction of the
\texorpdfstring{$\boldsymbol{\Phi^{-1/3}}$}{Phi(-1/3) Scaling}}
\label{app:phi_minus_one_third}

This appendix provides an explicit \emph{self-consistent closure}
(Class~0A) that yields the cubic-root scaling
$\delta t_{\min}\propto \Phi^{-1/3}$ for photon-limited temporal
resolution in diffusive broadening observed through finite-rate
localization measurements.

\paragraph{Hard guard against a common misreading.}
The $\Phi^{-1/3}$ exponent in this paper does \emph{not} follow from
minimizing a surrogate expression of the form
$2Dt+\sigma_m^2/(\Phi t)$, nor from any direct extremization in $t$.
Any derivation chain that starts from
$\sigma_{\mathrm{est}}^2(t)=2Dt+\sigma_m^2/(\Phi t)$ and then imposes
$\delta t\sim \sigma_{\mathrm{est}}^2(\delta t)/(2D)$ will algebraically
produce a $\Phi^{-1/2}$ scaling, i.e.\ a different inference-only regime.
We therefore \emph{do not use} that surrogate term anywhere in the
construction below.

Instead, the cubic-root scaling is obtained by closing a loop between:
(i) an explicit estimator class for the diffusive width,
(ii) its finite-sample variance under Poissonian photon counts, and
(iii) the operational decision criterion used throughout the paper.

\subsection{Setup and assumptions}

We consider one-dimensional Brownian motion with diffusion coefficient
$D$ and latent variance
\begin{equation}
\mathrm{Var}[x(t)] = 2Dt.
\end{equation}
The system is observed through a Poisson point process of detections
with constant photon flux $\Phi$. Each detection yields a noisy position
sample
\begin{equation}
y_k = x(t_k) + \varepsilon_k,
\qquad
\varepsilon_k \sim \mathcal{N}(0,\sigma_m^2),
\end{equation}
with independent Gaussian point-spread uncertainty $\sigma_m$.

Over an observation window of duration $t$, the number of detections
$N$ satisfies
\begin{equation}
N \sim \mathrm{Poisson}(\Phi t),
\qquad
\mathbb{E}[N]=\Phi t.
\end{equation}

We define the \emph{observable} per-sample variance (latent diffusion
plus measurement noise) as
\begin{equation}
V(t) \;:=\; \mathrm{Var}[y|t] \;=\; 2Dt + \sigma_m^2.
\label{eq:V_def}
\end{equation}

Assumptions:
\begin{enumerate}
\item Normal diffusion with finite $D$ and negligible drift.
\item Poissonian detection statistics with rate $\Phi$.
\item Independent Gaussian measurement noise with variance $\sigma_m^2$.
\item Local (quadratic) Fisher-information geometry for small parameter
offsets and a fixed decision threshold $D^\ast$
(Appendix~\ref{app:fisher_qfi}).
\item Class~0A closure: the effective observation window is identified
with the minimal resolvable time scale, producing a self-consistent loop.
\end{enumerate}

\subsection{Estimator class and finite-sample variance}

Consider the natural estimator of the observable variance $V(t)$ from
the $N$ samples:
\begin{equation}
\widehat{V}
\;=\;
\frac{1}{N}\sum_{k=1}^N (y_k-\bar y)^2.
\end{equation}
For Gaussian samples, $\widehat{V}$ has (to leading order in $1/N$) a
variance
\begin{equation}
\mathrm{Var}[\widehat{V}\,|\,N]
\;\sim\;
\frac{2V(t)^2}{N}.
\label{eq:var_Vhat_given_N}
\end{equation}

Time is inferred through $V(t)$ via Eq.~\eqref{eq:V_def}, i.e.
\begin{equation}
t = \frac{V(t)-\sigma_m^2}{2D},
\qquad
\widehat{t} = \frac{\widehat{V}-\sigma_m^2}{2D}.
\end{equation}
Propagation of uncertainty gives
\begin{equation}
\mathrm{Var}[\widehat{t}\,|\,N]
\;=\;
\frac{\mathrm{Var}[\widehat{V}\,|\,N]}{(2D)^2}
\;\sim\;
\frac{V(t)^2}{2D^2\,N}.
\label{eq:var_that_given_N}
\end{equation}

Averaging over Poissonian $N$ amounts, at scaling level, to replacing
$N$ by its typical size $\Phi t$ (this is sufficient for exponent
accounting and does not affect the closure class):
\begin{equation}
\mathrm{Var}[\widehat{t}]
\;\sim\;
\frac{V(t)^2}{2D^2\,\Phi t}.
\label{eq:var_that}
\end{equation}

\subsection{Self-consistent closure (Class 0A fixed point)}

The Class~0A construction identifies the operationally relevant window
with the resolvable scale itself:
\begin{equation}
\delta t^2 \;\sim\; \mathrm{Var}[\widehat{t}] \quad \text{evaluated at } t=\delta t.
\label{eq:self_consistency}
\end{equation}
Substituting Eq.~\eqref{eq:var_that} and setting $t=\delta t$ yields
\begin{equation}
\delta t^2
\;\sim\;
\frac{V(\delta t)^2}{2D^2\,\Phi\,\delta t}
\quad\Longrightarrow\quad
\delta t^3
\;\sim\;
\frac{V(\delta t)^2}{2D^2\,\Phi}.
\label{eq:fixed_point_balance}
\end{equation}

In the photon-limited temporal-localization regime relevant to this
construction, the closure occurs while $2D\,\delta t \ll \sigma_m^2$ so
that $V(\delta t)\approx \sigma_m^2$.
Then Eq.~\eqref{eq:fixed_point_balance} becomes
\begin{equation}
\delta t_\ast^3
\;\sim\;
\frac{\sigma_m^4}{2D^2\,\Phi},
\qquad
\delta t_\ast
\;\sim\;
\left(\frac{\sigma_m^4}{D^2\,\Phi}\right)^{1/3}.
\label{eq:t_star}
\end{equation}
Therefore,
\begin{equation}
\delta t_\ast \propto \Phi^{-1/3}.
\end{equation}

\subsection{Fisher information and the operational decision criterion}

For a Gaussian family with known mean and variance $V(t)$, the per-sample
Fisher information about $V$ is $\FI(V)=1/(2V^2)$.
By the chain rule with $V(t)=2Dt+\sigma_m^2$,
\begin{equation}
\FI_1(t)
\;=\;
\FI(V)\left(\frac{dV}{dt}\right)^2
\;=\;
\frac{1}{2V(t)^2}(2D)^2
\;=\;
\frac{2D^2}{V(t)^2}.
\end{equation}
For $N$ i.i.d.\ detections, Fisher information is additive, so
\begin{equation}
\FI_T(t)\;\sim\; N\,\FI_1(t)
\;\sim\;
\Phi t \cdot \frac{2D^2}{V(t)^2}
\;=\;
\frac{2D^2\,\Phi\,t}{(2Dt+\sigma_m^2)^2}.
\label{eq:fi_scaling}
\end{equation}

The operational resolution criterion used in the paper is
\begin{equation}
\delta t^2\,\FI_T(\delta t) \;\gtrsim\; 2D^\ast.
\label{eq:decision_again}
\end{equation}
Substituting Eq.~\eqref{eq:fi_scaling} into Eq.~\eqref{eq:decision_again}
and working in the same closure regime $2D\,\delta t \ll \sigma_m^2$
yields
\begin{equation}
\delta t^2 \cdot \frac{2D^2\,\Phi\,\delta t}{\sigma_m^4}
\;\gtrsim\;
2D^\ast
\quad\Longrightarrow\quad
\delta t_{\min}^3
\;\gtrsim\;
D^\ast\,\frac{\sigma_m^4}{D^2\,\Phi}.
\end{equation}
Thus,
\begin{equation}
\delta t_{\min}
\;\sim\;
\left(\frac{\sigma_m^4}{D^2\,\Phi}\right)^{1/3}
\;\propto\;
\Phi^{-1/3},
\end{equation}
up to inessential constants (including $D^\ast$).

\subsection{Interpretation and scope}

The cubic-root scaling here is a \emph{closure-limited construction}
(Class~0A):
finite photon flux limits how quickly time-information can be harvested,
and the resulting estimator variance feeds back into the effective
observation window through Eq.~\eqref{eq:self_consistency}.
This is not an inference-optimality statement.

By contrast, \emph{inference-only} treatments that do not identify the
observation window with the resolution scale (i.e.\ no closure loop)
typically produce shot-noise-type scaling such as $\Phi^{-1/2}$ in
appropriate regimes. That behavior is not contradicted by the present
construction; it belongs to a different model class.

\subsection{Connection to generalizations}

For anomalous transport with $\mathrm{MSD}(t)\sim C t^\alpha$, the same
closure logic yields $\delta t_{\min}\propto\Phi^{-1/(2+\alpha)}$, as
derived in Appendix~\ref{app:ctrw_derivations}. The normal diffusion
case corresponds to $\alpha=1$ and serves as the baseline.

% ============================================================
% End of appendices/A_phi_minus_one_third.tex
% ============================================================
