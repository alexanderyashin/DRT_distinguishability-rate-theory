% ============================================================
% appendices/B_ctrw_derivations.tex  (FULL REPLACEMENT)
% ============================================================

\section{Self-Consistent Fixed-Point Construction for Anomalous Transport:
\texorpdfstring{$\boldsymbol{\Phi^{-1/(2+\alpha)}}$}{Phi(-1/(2+alpha)) Scaling}}
\label{app:ctrw_derivations}

This appendix generalizes the Class~0A self-consistent closure mechanism
(appendix-level construction) to anomalous transport with mean-square
displacement (MSD) scaling $\mathrm{MSD}(t)\sim C\,t^\alpha$.
The resulting scaling
$\delta t_{\min}\propto \Phi^{-1/(2+\alpha)}$ is \emph{construction-based}
and \emph{not} inference validation.

\paragraph{Misreading guard.}
The exponent derived here is \emph{not} obtained by minimizing an explicit
objective in $t$ (no ``optimal observation time'' is claimed).
In particular, one must not interpret the construction as the extremum of a
surrogate ``variance'' expression of the form
$C t^\alpha + \sigma_m^2/(\Phi t)$.
Any such extremization belongs to a different modeling choice and can lead to
different exponents. The Class~0A result is defined by a closure loop on an
estimator family under a Poisson-limited channel.

\subsection{Setup and assumptions}

We assume a one-dimensional stochastic process with MSD scaling
\begin{equation}
\langle x^2(t)\rangle \sim C\,t^\alpha,
\qquad 0<\alpha<2,
\end{equation}
observed through a Poisson point process of detection events with constant
flux $\Phi$.
Each detection yields an independent position estimate with Gaussian
uncertainty $\sigma_m$.

Assumptions (same epistemic status as Appendix~A):
\begin{enumerate}
\item MSD scaling $\mathrm{MSD}(t)\sim C\,t^\alpha$ holds over the relevant
time range (local power-law regime).
\item Poissonian detection statistics with rate $\Phi$.
\item Independent per-detection measurement noise with variance $\sigma_m^2$.
\item A fixed operational decision threshold $D^\ast$ used internally
(Appendix~\ref{app:fisher_qfi}) to interpret ``resolvable'' as an operational
criterion, not as an inference-optimal bound.
\end{enumerate}

\subsection{Closure target: an estimator--variance loop}

As in Appendix~A, the construction is defined by a \emph{self-consistent loop}
on an estimator family.
Let $\hat t$ denote an elapsed-time estimator built from the measurement record
restricted to a window of duration $t$.
Denote its mean-square error (MSE) by
\begin{equation}
v_{\hat t}(t) \;:=\; \mathbb{E}\!\left[(\hat t - t)^2\right].
\end{equation}

The Class~0A closure postulates that the operational resolution scale is the
fixed point of this estimator variance:
\begin{equation}
\delta t \;\sim\; \sqrt{v_{\hat t}(\delta t)}.
\label{eq:ctrw_closure_loop}
\end{equation}
This is a \emph{closure condition}. It is not a CRLB extremization and does not
claim inference-optimality.

\subsection{Scaling of the estimator variance under Poisson-limited sampling}

Under Poisson-limited sampling, the number of detections in a window of length
$t$ satisfies $N\sim \mathrm{Poisson}(\Phi t)$, hence $\mathbb{E}[N]\asymp \Phi t$.
Finite sampling implies that any estimator built from the record inherits an
MSE contribution that scales inversely with the expected event count,
\begin{equation}
v_{\hat t}(t)\;\propto\;\frac{1}{\Phi t}\;\times\;\Big(\text{time-sensitivity factor}\Big)^{-2}.
\label{eq:ctrw_sampling_skeleton}
\end{equation}
The ``time-sensitivity factor'' is determined by how strongly the observable
statistics change with $t$.

For MSD scaling $\mathrm{MSD}(t)\sim C t^\alpha$, the characteristic spatial
scale grows as $\sqrt{C}\,t^{\alpha/2}$, so local time sensitivity of the scale
is
\begin{equation}
\frac{d}{dt}\left(t^{\alpha/2}\right)\;\asymp\; t^{\alpha/2-1}.
\end{equation}
Under the same local-geometry regime used throughout the paper (quadratic
approximation in small $\delta t$), this induces the scaling
\begin{equation}
v_{\hat t}(t)\;\propto\;\frac{\sigma_m^2}{\Phi t}\;\times\; t^{2-\alpha},
\label{eq:ctrw_vhat_scaling}
\end{equation}
up to dimensionless constants and factors absorbed into $C$.
Equation~\eqref{eq:ctrw_vhat_scaling} is the only scaling input needed for the
closure exponent.

\subsection{Fixed-point exponent}

Substituting Eq.~\eqref{eq:ctrw_vhat_scaling} into the closure loop
Eq.~\eqref{eq:ctrw_closure_loop} gives
\begin{equation}
\delta t
\;\sim\;
\sqrt{v_{\hat t}(\delta t)}
\;\propto\;
\sqrt{\frac{\sigma_m^2}{\Phi\,\delta t}\;\delta t^{2-\alpha}}
\;=\;
\frac{\sigma_m}{\sqrt{\Phi}}\;\delta t^{(1-\alpha/2)}.
\end{equation}
Rearranging yields the fixed-point scaling
\begin{equation}
\delta t^{1+\alpha/2}
\;\propto\;
\frac{\sigma_m}{\sqrt{\Phi}},
\end{equation}
or, equivalently,
\begin{equation}
\delta t_{\min}
\;\propto\;
\Phi^{-1/(2+\alpha)}.
\end{equation}
All prefactors depend on the chosen estimator family and the local-regime
constants; the exponent is the construction output.

\paragraph{Reduction to normal diffusion.}
For $\alpha=1$ the construction reduces to
$\delta t_{\min}\propto \Phi^{-1/3}$, consistent with Appendix~A and
Section~\ref{sec:classical_diffusion}, with the same epistemic status
(Class~0A closure).

\subsection{Epistemic status and relation to Class 0B simulations}

The exponent $-1/(2+\alpha)$ obtained here is epistemic and
construction-based.
It depends explicitly on the MSD exponent $\alpha$ and the Poisson-limited
observation channel.

Existing CTRW Monte Carlo demonstrations in the repository that start from
$\delta t_0\propto \Phi^{-1/(2+\alpha)}$ are Class~0B
(exponent-imposed generators) and serve only as pipeline consistency checks,
not as inference evidence.

% ============================================================
% End of appendices/B_ctrw_derivations.tex
% ============================================================
