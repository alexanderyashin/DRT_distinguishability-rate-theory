% ============================================================
% appendices/B_ctrw_derivations.tex  (FULL REPLACEMENT)
% ============================================================

\section{Anomalous Diffusion: CTRW Derivations}
\label{app:ctrw_derivations}

This appendix generalizes the self-consistent inference argument of
Appendix~\ref{app:phi_minus_one_third} to anomalous diffusion described by
continuous-time random walks (CTRW).
The purpose is to derive the universal photon-limited scaling
\[
\delta t_{\min}\propto \Phi^{-1/(2+\alpha)}
\]
under minimal, explicit, and operational assumptions.

\subsection{CTRW model and transport exponent}

We consider a CTRW characterized by independent spatial jumps with finite second
moment and waiting times drawn from a heavy-tailed distribution.
In the long-time limit, the mean-square displacement (MSD) scales as
\begin{equation}
\langle x^2(t)\rangle \;\sim\; C\, t^\alpha,
\qquad 0<\alpha<2,
\end{equation}
where $C$ is a generalized transport coefficient.
The regimes $\alpha<1$, $\alpha=1$, and $\alpha>1$ correspond to subdiffusion,
normal diffusion, and superdiffusion, respectively.

\paragraph{Assumptions.}
The derivation relies on:
\begin{enumerate}
\item A well-defined asymptotic MSD exponent $\alpha$.
\item Poissonian detection statistics with constant photon flux $\Phi$.
\item Independent measurement noise with variance $\sigma_m^2$ per detection.
\item Local inference governed by a fixed distinguishability threshold $D^\ast$
(Appendix~\ref{app:fisher_qfi}).
\end{enumerate}

\subsection{Operational localization balance}

As in Appendix~\ref{app:phi_minus_one_third}, the effective localization variance
$\sigma^2(t)$ reflects a balance between anomalous spreading and
measurement-induced localization.
Operationally,
\begin{equation}
\sigma^2(t)\;\sim\; C\, t^\alpha \;+\; \frac{\sigma_m^2}{\Phi t}.
\label{eq:ctrw_balance}
\end{equation}
The first term captures the anomalous transport of the latent trajectory, while
the second encodes finite-rate information acquisition through Poisson-limited
measurements.

\subsection{Self-consistent inference time scale}

The characteristic inference time scale is obtained by minimizing
$\sigma^2(t)$ with respect to $t$.
Differentiating Eq.~\eqref{eq:ctrw_balance} gives
\begin{equation}
\frac{d\sigma^2}{dt}
= \alpha C t^{\alpha-1}
- \frac{\sigma_m^2}{\Phi t^2}.
\end{equation}
Setting the derivative to zero yields the fixed point
\begin{equation}
t_\ast \;\sim\;
\left(\frac{\sigma_m^2}{C\,\Phi}\right)^{\!1/(2+\alpha)}.
\label{eq:ctrw_t_star}
\end{equation}
This scale marks the balance between anomalous spreading and finite-rate
measurement-induced uncertainty reduction.

\subsection{Temporal distinguishability}

Temporal distinguishability arises because the width of the anomalous diffusion
process depends on time.
For small temporal offsets $\delta t$, Fisher information about time scales
inversely with the localization variance,
\begin{equation}
I_T(t) \;\sim\; \frac{t}{\sigma^2(t)},
\label{eq:ctrw_fi_scaling}
\end{equation}
up to dimensionless constants.
This scaling reflects the fact that resolving time requires resolving changes in
the MSD over the observation interval.

Evaluating Eq.~\eqref{eq:ctrw_fi_scaling} at the self-consistent scale $t_\ast$
gives
\begin{equation}
I_T(t_\ast)
\;\sim\;
\Phi^{\frac{2}{2+\alpha}}\,
C^{-\frac{1}{2+\alpha}}\,
\sigma_m^{-\frac{2\alpha}{2+\alpha}},
\end{equation}
where numerical prefactors have been suppressed, as they do not affect scaling
exponents.

\subsection{Operational time resolution}

The minimal resolvable temporal offset $\delta t_{\min}$ follows from the
decision criterion
\begin{equation}
\delta t^2\, I_T \;\gtrsim\; 2D^\ast.
\end{equation}
Substituting the scaling of $I_T(t_\ast)$ yields
\begin{equation}
\delta t_{\min}
\;\sim\;
\Phi^{-\,\frac{1}{2+\alpha}},
\label{eq:ctrw_scaling}
\end{equation}
up to prefactors depending on $C$, $\sigma_m$, and $D^\ast$.

\subsection{Consistency with normal diffusion}

Setting $\alpha=1$ in Eq.~\eqref{eq:ctrw_scaling} recovers the cubic-root law
$\delta t_{\min}\propto \Phi^{-1/3}$ derived in
Appendix~\ref{app:phi_minus_one_third}.
The CTRW result therefore represents a continuous generalization rather than a
distinct mechanism.

\subsection{Numerical confirmation}

Monte Carlo simulations of CTRW trajectories with varying $\alpha$ confirm the
predicted scaling exponents.
An $\alpha$-sweep comparing measured slopes against the theoretical prediction
$-1/(2+\alpha)$ is included in the frozen Evidence Pack (Figure~9).
Agreement within statistical uncertainty supports the rate-limited
interpretation.

\subsection{Epistemic status and regime of validity}

The CTRW scaling is epistemic.
It reflects inference limits imposed by anomalous transport combined with
Poisson-limited observation and vanishes in the limit of unbounded information
rate.
Deviations from Eq.~\eqref{eq:ctrw_scaling} indicate either pre-asymptotic
regimes, violations of the assumed MSD scaling, or additional unmodeled
information channels.

\subsection{Falsifiability}

Observation of sustained scaling steeper than $\Phi^{-1/(2+\alpha)}$ under the
same CTRW and observation assumptions would falsify the rate-limited hypothesis
for anomalous diffusion.
Conversely, systematic deviations toward weaker scaling diagnose noise or model
misspecification rather than ontological constraints.

% ============================================================
% End of appendices/B_ctrw_derivations.tex
% ============================================================
