% ============================================================
% metadata.tex
% ============================================================
% Canonical metadata for the DRT paper.
% This file contains NO scientific content and MUST remain
% stable across revisions. It defines identity, scope, and
% bibliographic anchors of the paper.
%
% Project: Distinguishability-Rate Theory (DRT)
% Layer: OC Extensions, K_2
% Status: Evidence Pack PLATINUM (frozen)
% ============================================================

% ----------------------------
% Title block
% ----------------------------
\newcommand{\PaperTitle}{%
Distinguishability as a Physical Primitive:\\
Rate-Limited Inference, Temporal Resolution,\\
and the Boundary of Quantum Ontology
}

\newcommand{\PaperShortTitle}{%
Distinguishability as a Physical Primitive
}

% ----------------------------
% Authorship
% ----------------------------
\newcommand{\PaperAuthors}{%
Alexander Yashin
}

\newcommand{\PaperAffiliations}{%
Ontology of Continua (OC) Extensions, Level K$_2$\\
Independent Researcher
}

% ----------------------------
% Date
% ----------------------------
% Intentionally set at build time for traceability
\newcommand{\PaperDate}{\today}

% ----------------------------
% Keywords (for indexing)
% ----------------------------
\newcommand{\PaperKeywords}{%
distinguishability,
rate-limited inference,
temporal resolution,
classical diffusion,
continuous monitoring,
quantum Fisher information,
epistemic bounds,
ontological limits,
quantum metrology,
anomalous diffusion
}

% ----------------------------
% Abstract (canonical version)
% ----------------------------
\newcommand{\PaperAbstract}{%
Distinguishability limits are frequently interpreted as evidence for
fundamental temporal or energetic discreteness imposed by quantum mechanics.
In this work, we demonstrate that a broad and practically relevant class of
such limits is instead epistemic in origin, arising from rate-limited
statistical inference under noise, finite statistics, and continuous
observation.
We introduce an operational framework in which temporal resolution is governed
by the growth rate of distinguishability between nearby hypotheses rather than
by microscopic time quanta.
Within this framework, we derive universal scaling laws for temporal inference
in classical diffusion, anomalous transport, continuous monitoring, and open
quantum systems, including the characteristic $\Phi^{-1/3}$ scaling observed
in diffusion under Poisson detection.
We identify a sharp boundary separating inference-limited regimes from
irreducible ontological constraints, and show how apparent discreteness may
emerge without invoking fundamental time quantization.
Our results clarify the physical meaning of temporal bounds, provide explicit
falsifiability criteria, and establish distinguishability as a unifying
primitive for reasoning about time, measurement, and ontology.
}

% ============================================================
% End of metadata.tex
% ============================================================
